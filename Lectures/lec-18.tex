\section{Direct Product of \( R \)-modules}

\begin{definition}[\textbf{Direct Product of \( R \)-modules}]
	Fix \( R \) ring and a family \( \{M_i\}_{i\in I} \) of left \( R \)-submodules. The
	\underline{direct product} \( \prod_{i\in I}M_i \) is the \textcolor{blue}{group direct product},
	i.e. on Cartesian product
	\[
		\prod_{i\in I} M_i \rsa (a_i)_i +(b_i)_i = (a+b)_i
	\]
	with component-wise scalar multiplication:
	\[
		\lambda \cdot (a_i)_i = (\lambda a_i)_i \quad \forall \; \lambda \in R
	\]
	It is easy to check that \( \prod_{i\in I} M_i \) is indeed a left \( R \)-module.
\end{definition}

\begin{remark}
	\leavevmode
	\begin{enumerate}
		\item \( (a_i)_i \in \prod_{i\in I}M_i \) is an element in the direct product, can imagine it as
		      a \textcolor{blue}{big vector with the i-th entry is \( a_i \in M_i \)}.
		\item We have the following projection as the morphism of \( R \)-modules.
		      \[
			      \begin{aligned}
				      \pi_j : \prod_{i\in I} M_i & \to M_j     \\
				      (x_i)_i                    & \mapsto x_j
			      \end{aligned}
		      \]
	\end{enumerate}
\end{remark}

\begin{proposition}[\textbf{Universal Property of Direct Product of \( R \)-modules}]
	Given any left \( R \)-module \( M \) and morphism of \( R \)-modules \( f_i: M \to M_i \) for \(
	i\in I\), there exists a unique morphism of \( R \)-module \( f: M \to \prod_{i\in I} M_i \), s.t.
	\begin{equation}
		\label{eq:univ-prop-direct-prod-module}
		\pi_j \circ f = f_j \quad \forall \; j
	\end{equation}
\end{proposition}

\begin{proof}
	Basically Formula \ref{eq:univ-prop-direct-prod-module} says that
	\[
		f(u) = \bace{f_i(u)}_i
	\]
	which gives a explicit formula and thus uniqueness is clear. For existence, we need to show that
	the above definition \( f \) is actually a \textcolor{blue}{morphism of \( R \)-modules}. It is
	clear
	\[
		a(f_i(u))_i = a f(u) \xlongequal{?} f(au) = (f_i(au))_i = \bace{a f_i(u)}_i
	\]
	and the sum is basically the same thing. The morphism is basically inherit from the fact that \(
	f_i \) is a morphism of \( R \)-modules.
\end{proof}

\section{Direct Sum of \( R \)-modules}

\begin{definition}[\textbf{Direct Sum of \( R \)-modules}]
	Given a family of left \( R \)-modules \( (M_i)_{i\in I} \), the \underline{direct sum}
	\[
		\bigoplus_{i\in I} M_i \subseteq \prod_{i\in I} M_i
	\]
	is a submodule of \( \prod_{i\in I}M_i \), such that it is defined as:
	\[
		\bigoplus_{i\in I}M_i = \{(a_i)_{i\in I} \; | \; a_i = 0 \text{ for all but
			\textcolor{red}{finitely} many }i\}
	\]
	One can check that this is indeed a submodule.
\end{definition}

\begin{remark}
	For each \( j \in I \), we can define the dual notion of projection
	\[
		\begin{aligned}
			\alpha_j : M_j & \to \bigoplus_{i\in I} M_i               \\
			\alpha_j (u)   & = (u_i)_i                                \\
			               & \text{where } u_i = \begin{cases}
				                                     u, \text{ if } i = j \\
				                                     0, \text{ otherwise}
			                                     \end{cases}
		\end{aligned}
	\]
\end{remark}
Note that direct sum is actually a dual notion of direct product

\begin{proposition}[\textbf{Universal Property of Direct Sum}]
	Given any left \( R \)-module and family of \( R \)-module morphisms \( f_j : M_j \to N \), there
	exists a unique morphism of \( R \)-module
	\[
		f: \bigoplus_{i\in I}M_i \to N
	\]
	s.t.
	\begin{equation}
		\label{eq:univ-prop-direct-sum-module}
		f\circ \alpha_j = f_j \quad \forall \l j \in I
	\end{equation}
\end{proposition}

\begin{proof}
	The key point is that given any \( u = (u_i)_{i\in I} \in \bigoplus_{i\in I}M_i \), have
	\[
		u = \sum_{i\in I} \alpha_i (u_i)
	\]
	which makes sense because of finite sum since only \textcolor{blue}{finitely} many \( u_i \) are
	non-zero. If \( f \) satisfies formula \ref{eq:univ-prop-direct-sum-module}, then
	\[
		f(u) = \sum_{i\in I}f_i(u_i) \quad \forall \; u = (u_i)_i \in \bigoplus_{i\in I}M_i
	\]
	this gives us a explicit formula and thus yield uniqueness. For existence, we define \( f \) by
	this formula and then it is clear that
	\[
		f \circ \alpha_j = f_j \quad \forall \; j \in I
	\]
	It remains to show that \( f \) is a \textcolor{blue}{morphism} of \( R \)-modules, which is
	similar as what we done above for universal property of direct product.
\end{proof}

\begin{note}
	If \( I \) is finite, then
	\[
		\bigoplus_{i\in I} M_i = \prod_{i\in I} M_i
	\]
\end{note}

\section{Quotient \( R \)-modules}
Let \( M \) be a left \( R \)-modules and \( N \subseteq M \) an \( R \)-submodule, in particular:
\( N \) is a subgroup of \( M \) and as \( M \) is abelian, it is then automatically normal. Hence
we have a quotient group \( \qo{M}{N} \) which is also an \textcolor{blue}{abelian group}, with
morphism of abelian groups
\[
	\begin{aligned}
		\pi : M & \to \qo{M}{N} \\
		\pi(u)  & = u + N
	\end{aligned}
\]

\begin{claim}
	We can make \( \qo{M}{N} \) a left \( R \)-module, s.t. \( \pi \) is \textcolor{blue}{morphism} of
	\( R \)-modules.
\end{claim}
Define:
\[
	\begin{aligned}
		R \times \qo{M}{N} & \to \qo{M}{N}     \\
		(\lambda, u + N)   & \to \lambda u + N
	\end{aligned}
\]
To check that it is indeed \textcolor{blue}{well-defined}, see that
\[
	\underbrace{u + N = u' + N}_{\iff u - u' \in N} \implies \underbrace{\lambda u + N = \lambda u' +
		N}_{\iff \lambda u - \lambda u'= \lambda (u-u') \in N}
\]
which is ok since \( N \) is submodule and thus it is well-defined.

One can easily check that with this scalar multiplication, \( \qo{M}{N} \) becomes an \( R \)-module
and \( \pi \) is a morphism of \( R \)-module.

\begin{proposition}[\textbf{Universal Property of \( \qo{M}{N} \)}]
	If \( f: M \to M' \) is a morphism of left \( R \)-module, s.t. \( N \subseteq \ker(f) \), then
	there exists a unique morphism of \( R \)-modules \( \overline{f}: \qo{M}{N} \to M' \), s.t. the
	following diagram is commutative
	\[
		\begin{tikzcd}
			M \arrow[r, "f"] \arrow[d, "\pi"'] & M' \\
			M/N \arrow[ru, "\overline{f}"'] &
		\end{tikzcd}
	\]
	namely
	\[ \overline{f} \circ \pi = f \].
\end{proposition}

\begin{proof}
	We already know this at the level of \textcolor{blue}{abelian groups}:
	\[
		\overline{f} : \qo{M}{N} \to M', \quad u + N \mapsto f(u)
	\]
	It is only left to show it is indeed morphism of \( R \)-modules:
	\[
		\overline{f} (a\cdot (u+N)) \xlongequal{?} a \cdot \overline{f}(u+N) \quad \forall \; a\in R,
		u\in M
	\]
	and see that:
	\[
		\begin{aligned}
			\text{LHS} & = \overline{f} (au + N) = f(au)                                                        \\
			\text{RHS} & = \underbrace{af(u) = f(au)}_{f \text{ is morphism of } R- \text{module}} = \text{LHS}
		\end{aligned}
	\]
\end{proof}

\begin{theorem}[\textbf{First Isomprhism Theorem}]
	If \( f: M \to M' \) is a surjective morphism of \( R \)-modules, the universal property of \(
	\qo{M}{\ker(f)} \implies \) there exists a unique morphism of \( R \)-modules
	\[
		\begin{aligned}
			\qo{M}{\ker(f)}          & \to M'    \\
			\overline{u}=u + \ker(f) & \mapsto u
		\end{aligned}
	\]
	And this is an isomprhism of \( R \)-modules. It is clear that it is a \underline{bijection} as we
	know from the case of \textcolor{blue}{abelian groups}.
\end{theorem}

\subsection{Submodules of \( \qo{M}{N} \)}
\begin{proposition}
	If \( N \subseteq M \) is a submodule of a left \( R \)-module, then we have an
	\textcolor{red}{order preserving bijection} as follow:
	\[
		\begin{tikzcd}[column sep=8em]
			\left\{
			\begin{tabular}{c}
				\text{Submodules of} \\
				$\qo{M}{N}$
			\end{tabular}
			\right\}
			\arrow[r, shift left=1.5ex, "T \subseteq \qo{M}{N} \rsa \pi^{-1}(T)"]
			&
			\left\{
			\begin{tabular}{c}
				\text{Submodules of } M \\
				\text{containing } N
			\end{tabular}
			\right\}
			\arrow[l, shift left=1.5ex, "\pi(K) \leftsquigarrow N \subseteq K \subseteq M"]
		\end{tikzcd}
	\]
	where
	\[
		\pi: M \to \qo{M}{N}
	\]
\end{proposition}

\begin{proof}
	We know this for abelian groups, it is only left to check they maps \( R \)-modules to \( R
	\)-modules, namely
	\begin{enumerate}
		\item If \( T \subseteq \qo{M}{N} \) be \( R \)-submodule \( \implies \pi^{-1}(T) \subseteq M \)
		      is an \( R \)-submodule.
		\item If \( K \subseteq M \) be \( R \)-submodule \( \implies \pi(K) \subseteq \qo{M}{N} \) is an
		      \( R \)-submodule.
	\end{enumerate}
	and we can use some \textcolor{blue}{beautiful tricks} to prove it.
	\begin{enumerate}
		\item Have
		      \[
			      \pi^{-1}(T) = \ker\bace{M \xrightarrow{\pi} \qo{M}{N} \to \qo{\qo{M}{N}}{T}}
		      \]
		\item Have
		      \[
            \pi(K) = \im\bace{K \hookrightarrow M \longrightarrow \qo{M}{N}}
		      \]
	\end{enumerate}
\end{proof}

\begin{remark}
	If \( N \subseteq K \subseteq M \implies \)
  \[
    \begin{tikzcd}[row sep=large, column sep=large]
        K \arrow[r] \arrow[d, equal] & M \arrow[r] & \qo{M}{N} \\
        K \arrow[rr] & & \pi(K) \arrow[u, phantom, "\rotatebox{90}{$\subseteq$}"]
    \end{tikzcd}
 \]
 and by first isomorphism theorem, we have 
 \[
   \pi (K) \cong\; \qo{K}{N}
 \] 
\end{remark}

\begin{theorem}[\textbf{Third Isomorphism Theorem}]
  If \( N \subseteq K \subseteq M \) be an \( R \)-submodule of \( M \), then:
  \[
    \qo{\qo{M}{N}}{\qo{K}{N}} \cong \; \qo{M}{K}
  \] 
\end{theorem}

\begin{proof}
  Consider
  	\[
		\begin{tikzcd}
      M \arrow[r] \arrow[d] & \qo{M}{K} \\
      \qo{M}{N} \arrow[ru, "\vp"'] &
		\end{tikzcd}
	\]
  where 
  \[
    \begin{aligned}
      \qo{M}{N} & \xrightarrow{\vp} \qo{M}{K} \\ 
          u + N &\mapsto u + K
    \end{aligned}
  \] 
  see that \( \vp \) is surjective, with
  \[
    \ker(\vp) = \qo{K}{N}
  \] 
  and the first isomorphism theorem yields the result.
\end{proof}

\begin{exercise}
  If \( M,N \) are left \( R \)-modules, have
  \[
    \Hom_{R} (M, N) = \{f : M \to N \; | \; f \text{ is morphism of } R- \text{modules}\}
  \]
  \textcolor{blue}{What structure does \( \Hom_{R}(M,N) \) have? What operations can you put on it?}
\end{exercise}
The answer is that it is always an abelian group, sometimes an \( R \)-module (if \( R \) is
\textcolor{blue}{commutative}!)

