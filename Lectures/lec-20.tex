\begin{notation}
	If \( I \) be a set and \( R \) be a ring, then:
	\[
		R^{(I)} \coloneqq \bigoplus_{i\in I}M_i \quad M_i = R \; \forall \; i
	\]
	and
	\[
		e_i \coloneqq (0, \ldots, 0, \underbrace{1}_{i}, 0, \ldots, 0)
	\]
\end{notation}
By combining the universal property of direct sums and the description of \( R \)-module morphism \(
R \to N\), implies the \textbf{Universal Property} of \( R^{(I)} \): \( \forall \; R  \)-module $N$,
and \( \forall \; (u_i)_{i\in I}, u_i \in N \; \forall \; i \), \( \exists \; ! \; R \) -module
homomorphism \( f: R^{(I)} \to N \), s.t. \( f(e_i) = u_i \; \forall \; i \). Basically we only need
to specify where \( e_i \) goes to secure the whole morphism.

\begin{note}
	\leavevmode
	\begin{itemize}
		\item
		      \[
			      \begin{aligned}
				      \im(f) & = \brac{\sum a_i u_i \; | \; a_i \in R, \text{ only finitely many } \ne 0} \\
				             & = \text{submodule } \angl{\brac{u_i \; | \; i \in I}}                      \\
				             & = \sum_{i\in I} R u_i
			      \end{aligned}
		      \]
		      \( \implies \{u_i \; | \; i \in I\} \) is system of generator of \( N \) iff \( f \) is
		      surjective.
		\item \( f \) injective iff the \( u_i, i\in I \) are \textbf{linearly independent}.
		\item \( f \) is isomorphism iff the \( u_i, i\in I \) form a \textbf{basis} of \( N \).
		\item In particular, \underline{a module is free} iff it is \textbf{isomorphic to some} \( R^{(I)} \).
	\end{itemize}
\end{note}

\begin{theorem}[\textbf{Linear Algebra Facts}]
	\leavevmode
	\begin{itemize}
		\item Every vector space over a field has a basis.
		\item Moreover, any basis have the same cardinality.
	\end{itemize}
\end{theorem}
When over a ring \( R \):
\begin{itemize}
	\item We saw that many \( R \)-modules are \textbf{not free}.
	\item If \( R \) commutative, then can reduce modulo a maximal ideal to show that any \( 2 \)
	      basis of a free left \( R \)-module have the same cardinality.
	\item If \( R \) is not commutative, then it can happen that
	      \[
		      R \cong \; R \oplus R
	      \]
	      as left \( R \)-module.
\end{itemize}
The above statements will be proved in \textbf{Practice Set 2}, and supply here later.

\section{Nakayama's Lemma}
Basically Nakayama's Lemma gives us some idea on how to relate module of arbitrary ring to a module
of a field.

\begin{lemma}[\textbf{Nakayama's Lemma}]
	\label{lem:nakayama}
	Let's fix \( R, \underline{m} \) a local ring being \( R \ne \{0\} \) is commutative and the only
	maximal ideal is \( \underline{m} \). If \( M \) is a finitely generated \( R \)-module, such that
	\( M = \underline{m} M \), then \( M = \{0\} \).
\end{lemma}
One may wonder the \textcolor{blue}{significance of Nakayama's Lemma}, basically it gives us a
functor
\[
	\begin{aligned}
		F: R-\underline{\text{mod}} & \to \qo{R}{\underline{m}}-\underline{\text{mod}} \\
		F(M)                        & = \qo{M}{\underline{m}M}
	\end{aligned}
\]
While we lose some information when applying \( F \), the lemma says that:
\[
	F(M) = \{0\} \implies M = \{0\}
\]
thus we can recover some of the information.

\begin{corollary}
	\label{cor:nakayama}
	If \( M \) is finitely generated over \( R \), and \( (R,\underline{m}) \) is local, have \( N
	\subseteq M \) be submodule, then \( M = N + \underline{m}M \implies M = N \).
\end{corollary}

\begin{proof}
	Apply \textbf{Nakayama's Lemma} to \(\overline{M} = \qo{M}{N} \), which is finitely generated as
	well, have:
	\[
		\begin{aligned}
			\underline{m} \cdot \qo{M}{N} & = \qo{\bace{\underline{m}M + N}}{N}  = \qo{M}{N} \\
			\implies \qo{M}{N} = \{0\}    & \implies M = N
		\end{aligned}
	\]
	where \( \underline{m} M + N \) being the smallest submodule containing \( \underline{m}M \) and
	\( N \).
\end{proof}

\begin{corollary}
	If \( M \) is finitely generated over \( R \), and \( (R, \underline{m}) \) being local ring, then
	\( u_1, \ldots, u_n \in M \) is a system of generators iff \( \overline{u_1}, \ldots,
	\overline{u_n} \in \qo{M}{\underline{m}M}\) is a system of generators over \( \qo{R}{\underline{m}} \).
\end{corollary}

\begin{proof}
	The ``only if'' part is clear. Now suppose that \( \overline{u_1}, \ldots, \overline{u_n} \in
	\qo{M}{\underline{m}M} \) is a system of generators. Let \( N = \sum_{i=1}^n R u_i \) be submodule
	of \( M \), thus system of generators gives us:
	\[
		M = \underline{m} M + N \xRightarrow{\text{Cor \ref{cor:nakayama}}} N = M
	\]
	thus \( u_1, \ldots, u_n \) is system of generators of \( M \).
\end{proof}

\begin{proof}[\textbf{Proof of Nakayama's Lemma \ref{lem:nakayama}}]
	Let's choose \( u_1, \ldots, u_n \) be system of generators of \( M \). Have \( M = \underline{m}M
	\implies \)
	\[
		\forall \; i, u_i = \sum_{j=1}^{d_i} a_{ij} v_{ij} \quad a_{ij} \in \underline{m}, v_{ij} \in M
	\]
	and can write
	\[
		v_{ij} = \sum_{k=1}^n b_{ijk} u_k \quad b_{ijk} \in R
	\]
	thus
	\[
		\begin{aligned}
			u_i & = \sum_{j=1}^{d_i} a_{ij} \sum_{k=1}^n b_{ijk} u_k                            \\
			    & = \sum_{k=1}^n c_{ik} u_k \qquad c_{ik} = \sum_{j=1}^{d_i} a_{ij} b_{ijk} \in
			\underline{m}
		\end{aligned}
	\]
	using the matrix notation, let \( C = (c_{ij})_{1\leq i, j \leq n} \), have:
	\[
		(I_n - C) \cdot
		\begin{pmatrix}
			u_1    \\
			\vdots \\

			u_n
		\end{pmatrix} = 0
	\]
	Multiply on the left by the classical adjoint of \( (I_n - C) \), yields that:
	\[
		\det(I_n - C) \cdot I_n \cdot
		\begin{pmatrix}
			u_1    \\
			\vdots \\
			u_n
		\end{pmatrix} = 0 \implies \det(I_n - C) \cdot u_i = 0 \; \forall \; i
	\]
	Notice that:
	\[
		\det (I_n - C) =
		\begin{vmatrix}
			1-c_{11} & - c_{12} & \cdots & -c_{1n}  \\
			-c_{21}  & 1-c_{22} & \cdots & -c_{2n}  \\
			         &          & \ddots &          \\
			-c_{n 1} & -c_{n 2} & \cdots & 1-c_{nn}
		\end{vmatrix} = 1 + \alpha \text{ for some } \alpha \in \underline{m}
	\]
	Since \( (R, \underline{m}) \) is local and \( \alpha \in \underline{m} \) with \( \underline{m} \)
	contains all it's invertible elements, thus \( 1 + \alpha \) is invertible. \todo{Check proof here.}
	Thus \( u_i = 0 \; \forall \; i \implies M = \{0\} \).
\end{proof}

\begin{definition}[\textbf{Noetherian (Artinian) Module}]
	\( M \) is a \underline{Noetherian (Artinian)} \( R \)-module if \( \not\exists \) infinitely
	\textbf{strictly increasing (resp. decreasing)} sequence of submodules of \( M \).
\end{definition}

\begin{note}
	By definition, \( R \) is left Noetherian/Artinian iff \( R \) is Noetherian/Artinian as left \( R \)-module.
\end{note}

\begin{proposition}
	For a left \( R \)-module \( M \), \textbf{TFAE}:
	\begin{enumerate}
		\item \( M \) is Noetherian.
		\item Every non-empty family of submodules of \( M \) contains a maximal element w.r.t.
		      inclusion.
		\item Every submodule of \( M \) is finitely generated.
	\end{enumerate}
\end{proposition}

\begin{proof}
	Follows the same word by word as for the case \( M = R \). See the proof for \textbf{Proposition}
	\ref{prop:noetherian-equiv}.
\end{proof}
