\begin{note}
  If \( I \) is an ideal in \( R \), then all ideals in \( \qo{R}{I} \) are of the form \( \qo{P}{I} \) where \( I \subseteq P \) is an ideal. See that by Isomorphism theorem:
  \[
    \qo{\qo{R}{I}}{\qo{P}{I}} \cong \; \qo{R}{P}
  \] 

  Hence \( \qo{P}{I} \) is prime ideal if and only if \( P \) is prime ideal.
\end{note}

\begin{eg}
  The following are equivalent:
  \begin{itemize}
    \item \( R \) is a domain.
    \item \( (x):= \{xf \; | \; f\in R[X]\} \) inside \( R[X] \) is a prime ideal.
  \end{itemize}

  \begin{proof}
    This follows if we show the following, which gives the result be \textbf{Proposition} \ref{prop:domain-qo-prime}:
    \[
      \qo{R[x]}{(x)} \cong \; R 
    \] 

    Consider the \( R \)-algebra homomorphism:
    \[
      \begin{aligned}
        R[x] &\xrightarrow{\vp} R \\ 
        \vp(x) &= 0 \\ 
        a_0 + a_1x + \cdots + a_nx^n &\to a_0 
      \end{aligned}
    \] 

    This is a surjective homomorphism, with kernel being:
    \[
      \ker(\vp) = (x)
    \] 

    Then by isomomorphism theorem \ref{thm:fundemental-isomorphism}, yields the result.
  \end{proof}
\end{eg}

\begin{question}
  What about now consider \( (x-a) \subseteq R[x] \)?
\end{question}

\begin{note}
  There exists a \( R \)-Algebra isomorphism:
  \[
    \begin{aligned}
      f: R[x] &\to R[x] \\ 
      f(x) &= (x-a)
    \end{aligned}
  \] 

  So see that \( (x) \) is prime ideal if and only if \( (x-a) \) is prime ideal, if and only if \( R \) is a domain, thus if and only if \( (x-a) \) is also a prime ideal.

  By the universal property of \( R[x] \): there exists a unique such morphism \( f \) of \( R \)-Algebra, and exists a unique morphism of \( R \)-Algebra \( R[X] \xrightarrow{g} R[X], \; x\mapsto x+a \), thus \( g = f^{-1} \), and we can use the universal property to show that the composition \( f\circ g \) and \( g\circ f \) are identity, which yields isomorphism property.
\end{note}

\subsection{Maximal Ideals}

\begin{definition}[\textbf{Maximal Ideal}]
  An ideal \( M \subseteq R \) is a \underline{maximal ideal} if:
  \begin{enumerate}
    \item \( M \ne R \).
    \item If \( M \subseteq I \subseteq R \) and \( I \) be an ideal, then \( I = M \) or \( I = R \).
  \end{enumerate}
\end{definition}

\begin{lemma}
  \label{lem:maximal-ideal-field}
  If \( R \) is a commutative ring, then \( R \) is a field if and only if \( \{0\} \) is a maximal ideal.
\end{lemma}

\begin{proof}
  \leavevmode
  \begin{itemize}
    \item Suppose that \( R \) is a field, then \( R \ne \{0\} \). If \( I \subseteq R \) is an ideal, and \( I \ne \{0\} \). Let \( a\in I \backslash\{0\} \), since \( R \) is a field, see that \( a \) is \textbf{invertible}. Then:
      \[
        \forall \; b\in R, \; \; b = (ba^{-1})a \in I \implies I = R 
      \] 
    \item If \( \{0\} \) is a maximal ideal, then \( R \ne \{0\} \). \( \forall \; a\in R \) with \( a \ne 0 \), then \( a\in (a) \ne \{0\} \implies  (a) = R \implies \exists \; b \in R \), s.t. \( ab=1 \implies a\) is invertible.
  \end{itemize}
\end{proof}

\begin{corollary}
  \label{cor:maximal-field}
  An ideal \( M \subseteq R \) is maximal if and only if \( \qo{R}{M} \) is a field.
\end{corollary}

\begin{proof}
  By correspondance between ideals of \( \qo{R}{M} \) and ideals in \( R \) \textbf{containing} \( M \), this follows from \textbf{Lemma} \ref{lem:maximal-ideal-field}.
\end{proof}

\begin{corollary}
  Every maximal ideal is prime ideal.
\end{corollary}

\begin{proof}
  This follows since \textbf{every field is a domain}. And by \textbf{Corollary} \ref{cor:maximal-field} and \textbf{Proposition} \ref{prop:domain-qo-prime}.
\end{proof}

\begin{eg}
  The following are ideals that are prime but not maximal:
  \begin{enumerate}
    \item \( \{0\} \subseteq \ZZ \) is a prime ideal, but not maximal ideal.
    \item \( (x) \subseteq \ZZ[x] \) is a prime ideal, but not maximal ideal.\todo{why?}
  \end{enumerate}
\end{eg}

\begin{theorem}
  \label{thm:maximal}
  If \( I \subsetneq R \) is a proper ideal in a commutative ring \( R \), then there exists \( M \) being a maximal ideal, s.t. \( I \subseteq M \).
\end{theorem}

To prove it we'll need the famous \textbf{Zorn's Lemma}.

\begin{lemma}[\textbf{Zorn's Lemma}]
  \label{lem:zorn}
  If \( (A, \leq) \) is a non-empty partially ordered set, s.t. every totally ordered subset \( B \subseteq A \) has an upper bound in \( A \) (\( \exists \; a\in A \), s.t. \( b\leq a \; \forall \; b \in B \)), then \( A \) has at least a maximal element. (\( \exists \; a\in A \), s.t. if \( a\leq a' \in A \implies a = a' \))
\end{lemma}

\begin{proof}[\textbf{Proof uses Zorn's Lemma \ref{lem:zorn}}]
  \todo{proof structure}
  Fix \( I \) be in the theorem, and let \( \cJ = \{J \subseteq R \text{ is ideal} \; | \; I \subseteq J\} \). See that it is ordered by inclusion: \( J \leq J' \iff J \subseteq J' \). Note that \( \cJ \ne \emptyset \) since \( I \in \cJ \).

  So our basic task is to check it satisfies the hypothesis in \textbf{Zorn's Lemma} \ref{lem:zorn}.

  Let \( \cJ' \subseteq \cJ \) be a totally orderded subset: namely if \( J_1, J_2 \in \cJ' \implies (J_1 \subseteq J_2) \lor (J_2 \subseteq J_1) \). Now let:
  \[
    J \coloneqq  \bigcup_{J' \in \cJ'} J' 
  \] 

  The \textbf{key point} is that \( J \) is an ideal. Suppose \( a,b \in J \), let \( J', J'' \in \cJ' \) are s.t. \( a\in J', \; b\in J'' \). If \( J'\subseteq J''\implies a \in J''\implies a+b \in J'' \implies a+b \in J \). Similarly for \( J'' \subseteq J' \).

  If \( x\in J \) and \( \lambda \in R \), then there exists \( J' \in \cJ' \), s.t. \( x\in J' \implies \lambda x \in J' \subseteq J \).

  See that \( J \ne \emptyset \) since \( 0 \in J \). So the \textbf{conclusion} is \( J \) is an ideal. And it is clear that \( I \subseteq J \).

  See that \( J \ne R \), otherwise \( 1\in J \implies 1 \in J' \) for some \( J' \in \cJ' \), contradict to the fact that \( J' \ne R \).

  It is clear that \( J' \leq J \; \forall \; J' \in \cJ' \implies J \) is the upperbound for \( \cJ' \). The Apply \textbf{Zorn's Lemma} \ref{lem:zorn}: there exists \( M \in \cJ \) to be the maximal element, and this is a maximal ideal that containing \( I \).
\end{proof}

\begin{corollary}
  If \( R\ne \{0\} \) is a commutative ring, then there exists a maximal ideals in \( R \), in particular, there exists a prime ideal.
\end{corollary}

\begin{proof}
  Apply the theorem with \( I = \{0\} \) shows the existence of maximal ideal.
\end{proof}

\section{Local Ring}

In this section we shall discuss local rings. \todo{leave some overview!}

\begin{definition}[\textbf{Local Ring}]
  A commutative ring \( R \) is a \underline{local ring} if \( R \) has a \textbf{unique} maximal ideal.
\end{definition}

\begin{proposition}
  For a commutative ring \( R \), the following are equivalent:
  \begin{enumerate}
    \item \( R \) is a local ring. (with maximal ideal \( M = \{a \in R \; | \; a \text{ is not invertible}\} \))
    \item \( R \ne \{0\} \) and for all \( a,b \in R \), s.t. \( a+b = 1 \), either \( a \) or \( b \) is invertible.
  \end{enumerate}
\end{proposition}

\begin{proof}
  Suppose that \( R \) is a local ring with maximal ideal \( M \), then \( M \subseteq \{a \in R \; | \; a \text{ is not invertible}\} \) since \( M \ne R \). If \( a\in R \) is not invertible, then \( (a) \ne R \), by \textbf{Theorem} \ref{thm:maximal}, it is contained in a maximal ideal \( \implies (a) \subseteq M \implies a \in M \).

  In this case, \( R \ne \{0\} \) since \( M \ne R \). If \( a+b = 1 \), then either \( a\not\in M \) or \( b \not \in M \implies a\) is invertible or \( b \) is invertible.
\end{proof}
