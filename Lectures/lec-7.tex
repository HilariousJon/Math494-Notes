\begin{note}
	If \( I \) is an ideal in \( R \), then all ideals in \( \qo{R}{I} \) are of the form \( \qo{P}{I} \)
	where \( I \subseteq P \) is an ideal. See that by isomorphism theorem:
	\[
		\qo{\qo{R}{I}}{\qo{P}{I}} \cong \; \qo{R}{P}
	\]

	Hence \textcolor{blue}{\( \qo{P}{I} \) is prime ideal if and only if \( P \) is prime ideal}.
\end{note}

\begin{eg}
	The following are equivalent:
	\begin{itemize}
		\item \( R \) is a domain.
		\item \( (x):= \{xf \; | \; f\in R[x]\} \) inside \( R[x] \) is a prime ideal.
	\end{itemize}

	\begin{proof}
		This follows if we show the following, which gives the result be \textbf{Proposition} \ref{prop:domain-qo-prime}:
		\[
			\qo{R[x]}{(x)} \cong \; R
		\]

		Consider the \( R \)-algebra homomorphism:
		\[
			\begin{aligned}
				R[x]                         & \xrightarrow{\vp} R \\
				\vp(x)                       & = 0                 \\
				a_0 + a_1x + \cdots + a_nx^n & \to a_0
			\end{aligned}
		\]

		This is a surjective homomorphism, with kernel being:
		\[
			\ker(\vp) = (x)
		\]

		Then by isomomorphism theorem \ref{thm:fundemental-isomorphism}, yields the result.
	\end{proof}
\end{eg}

\begin{question}
	What about now consider \( (x-a) \subseteq R[x] \)?
\end{question}

\begin{note}
	There exists a \( R \)-Algebra isomorphism:
	\[
		\begin{aligned}
			f: R[x] & \to R[x] \\
			f(x)    & = x-a
		\end{aligned}
	\]

	So see that \( (x) \) is prime ideal if and only if \( (x-a) \) is prime ideal, if and only if
	\( R \) is a domain, thus if and only if \( (x-a) \) is also a prime ideal.

	By the universal property of \( R[x] \): there exists a unique such morphism \( f \) of
	\( R \)-Algebra, and exists a unique morphism of \( R \)-Algebra
	\( R[X] \xrightarrow{g} R[X], \; x\mapsto x+a \), thus \( g = f^{-1} \), and we can use the
	universal property to show that the composition \( f\circ g \) and \( g\circ f \) are identity,
	which yields isomorphism property.
\end{note}

\begin{definition}[\textbf{Comaximal Ideal}]
	Given \( R \) to be a commutative ring, let \( I_1, I_2 \) be two ideal s.t. \( I_1 \ne I_2 \), these two ideals are \underline{comaximal} if:
	\[
		I_1 + I_2 = R
	\]
\end{definition}

\begin{proposition}
	When \( I_1 + I_2 = R\), namely when they are comaximal, we have:
	\[
		I_1 \cap I_2 = I_1 I_2
	\]
\end{proposition}

\begin{proof}
	The ideal \( I_1 I_2 \) always contained in \( I_1 \cap I_2 \). Since there exists \( 1 = x+y \) for some \( x \in I_1, y\in I_2 \) since \( I_1 + I_2 = R \). Thus for \( c\in I_1 \cap I_2 \) we have \( c = c1 = cx + cy \in I_1 I_2 \) thus \( I_1 \cap I_2 \subseteq I_1 I_2 \).
\end{proof}

\begin{theorem}[\textbf{Generalized Chinese Remainder Theorem}]
	Let \( I_1, \ldots, I_n \) be ideals in \( R \) such that \( I_i + I_j = R \) for all \( i\ne j \). Then:
	\[
		\qo{R}{I_1 \cap \cdots \cap I_n} \cong \; \prod_{i=1}^n \qo{R}{I_i}
	\]
	with the ring homomorphism defined by:
	\[
		\begin{aligned}
			R & \to \qo{R}{I_1} \times \cdots \times \qo{R}{I_n} \\
			r & \mapsto (r+I_1,\ldots, r+I_n)
		\end{aligned}
	\]
	with kernel being \( I_1 \cap I_2 \cap \cdots \cap I_n \).
\end{theorem}

\begin{proof}[\textbf{Sketch}]
	One should first prove the case that when \( I_1 + I_2 = R \), then:
	\[
		\qo{R}{I_1 \cap I_2} \cong \; \qo{R}{I_1} \times \qo{R}{I_2}
	\]
	by first isomorphism theorem. Then show that:
	\[
		I_1 + (I_2 \cap I_3) = R
	\]
	by noticing that one can have such decomposition:
	\[
		\begin{aligned}
			1        & = (x+y) \cdot(u+v)                                                       \\
			         & = xu + yu + xv + yv                                                      \\
			         & \text{where } x,u\in I_1, y\in I_2, v \in I_3                            \\
			\implies & xu \in I_1, yu\in I_1 \cap I_2, xv \in I_1 \cap I_3, yv \in I_2 \cap I_3 \\
			         & \text{with } xu + yu + xv \in I_1, yv \in I_2 \cap I_3
		\end{aligned}
	\]
	and proceed on induction on \( n \).
\end{proof}

\subsection{Maximal Ideals}

\begin{definition}[\textbf{Maximal Ideal}]
	An ideal \( M \subseteq R \) is a \underline{maximal ideal} if:
	\begin{enumerate}
		\item \( M \ne R \).
		\item If \( M \subseteq I \subseteq R \) and \( I \) be an ideal, then \( I = M \) or \( I = R \).
	\end{enumerate}
\end{definition}
Notably, it is not necessarily for a ring to poccess a maximal ideal unless \( 1_R \ne 0_R \).
One can make such ring by taking abelian groups that pocess no maixmal subgroup, for example
\( \QQ \), and just set the multiplication into the trivial case \( ab = 0 \; \forall \; a,b\).
\textbf{Theorem} \ref{thm:maximal} state that when \( 1_R \ne 0_R \) there's always
maximal ideal. Hence any result related to maximal ideal forces the ring to be non-zero ring.

\begin{lemma}
	\label{lem:maximal-ideal-field}
	If \( R \) is a commutative ring, then \( R \) is a field if and only if \( \{0\} \) is a maximal
	ideal.
\end{lemma}

\begin{proof}
	\leavevmode
	\begin{itemize}
		\item Suppose that \( R \) is a field, then \( R \ne \{0\} \). If \( I \subseteq R \) is an ideal, and \( I \ne \{0\} \). Let \( a\in I \backslash\{0\} \), since \( R \) is a field, see that \( a \) is \textbf{invertible}. Then:
		      \[
			      \forall \; b\in R, \; \; b = (ba^{-1})a \in I \implies I = R
		      \]
		\item If \( \{0\} \) is a maximal ideal, then \( R \ne \{0\} \). \( \forall \; a\in R \) with \( a \ne 0 \), then \( a\in (a) \ne \{0\} \implies  (a) = R \implies \exists \; b \in R \), s.t. \( ab=1 \implies a\) is invertible.
	\end{itemize}
\end{proof}

\begin{corollary}
	\label{cor:maximal-field}
	An ideal \( M \subseteq R \) is maximal if and only if \( \qo{R}{M} \) is a field.
\end{corollary}

\begin{proof}
	By correspondance between ideals of \( \qo{R}{M} \) and ideals in \( R \) \textbf{containing} \( M \), this follows from \textbf{Lemma} \ref{lem:maximal-ideal-field}.
\end{proof}
In particular this corollary basically tells us how to construct some fields, we later shall use it to construct all kinds of finite fields, by taking quotients on the ring \( \ZZ[x] \). To see whether an ideal is maximal, either try to find some ideal strictly bigger than it or try to prove the ring quotient it out becomes a field.

\begin{eg}
	\leavevmode
	\begin{itemize}
		\item Let \( R \) be the ring of all functions from \( [0,1] \to \RR \), see that for each \( a \in [0,1] \) define:
		      \[
			      M_a = \{ f \in R \; | \; f(a) = 0\}
		      \]
		      and since evaluation is a surjective ring homomoprhism from \( R \to \RR \), thus:
		      \[
			      \qo{R}{M_a} \cong \; \RR
		      \]
		      and thus \( M_a \) being a maximal ideal.
		\item \( (x) \in \ZZ[x] \) is not a maixmal ideal. One can see that from either \( (x) \subsetneq (2,x) \subseteq \ZZ[x] \) or \( \qo{\ZZ[x]}{(x)} \cong\; \ZZ \) with \( \ZZ \) not being a field.
	\end{itemize}
\end{eg}

\begin{corollary}
	Every maximal ideal is prime ideal.
\end{corollary}

\begin{proof}
	This follows since \textbf{every field is a domain}. And by \textbf{Corollary} \ref{cor:maximal-field} and \textbf{Proposition} \ref{prop:domain-qo-prime}.
\end{proof}

\begin{eg}
	The following are ideals that are prime but not maximal:
	\begin{enumerate}
		\item \( \{0\} \subseteq \ZZ \) is a prime ideal, but not maximal ideal, same ideal as the proof
		      below.
		\item \( (x) \subseteq \ZZ[x] \) is a prime ideal, but not maximal ideal. Since
		      \[
			      \qo{\ZZ[x]}{(x)} \cong \; \ZZ
		      \]
		      with \( \ZZ \) being an integral domain not a field, thus \( (x) \) is prime ideal but not
		      maximal ideal.
	\end{enumerate}
\end{eg}

\begin{theorem}
	\label{thm:maximal}
	If \( I \subsetneq R \) is a proper ideal in a commutative ring \( R \), then there exists \( M \) being a maximal ideal, s.t. \( I \subseteq M \).
\end{theorem}

To prove it we'll need the famous \textbf{Zorn's Lemma}.

\begin{lemma}[\textbf{Zorn's Lemma}]
	\label{lem:zorn}
	If \( (A, \leq) \) is a non-empty partially ordered set, s.t. every totally ordered subset \( B \subseteq A \) has an upper bound in \( A \) (\( \exists \; a\in A \), s.t. \( b\leq a \; \forall \; b \in B \)), then \( A \) has at least a maximal element. (\( \exists \; a\in A \), s.t. if \( a\leq a' \in A \implies a = a' \))
\end{lemma}

\begin{proof}[\textbf{Proof uses Zorn's Lemma \ref{lem:zorn}}]
	Fix \( I \) be in the theorem, and let \( \cJ = \{J \subseteq R \text{ is ideal} \; | \; I \subseteq J\} \). See that it is ordered by inclusion: \( J \leq J' \iff J \subseteq J' \). Note that \( \cJ \ne \emptyset \) since \( I \in \cJ \).

	So our basic task is to check it satisfies the hypothesis in \textbf{Zorn's Lemma} \ref{lem:zorn}.

	Let \( \cJ' \subseteq \cJ \) be a totally orderded subset: namely if \( J_1, J_2 \in \cJ' \implies (J_1 \subseteq J_2) \lor (J_2 \subseteq J_1) \). Now let:
	\[
		J \coloneqq  \bigcup_{J' \in \cJ'} J'
	\]

	The \textbf{key point} is that \( J \) is an ideal. Suppose \( a,b \in J \), let \( J', J'' \in \cJ' \) are s.t. \( a\in J', \; b\in J'' \). If \( J'\subseteq J''\implies a \in J''\implies a+b \in J'' \implies a+b \in J \). Similarly for \( J'' \subseteq J' \).

	If \( x\in J \) and \( \lambda \in R \), then there exists \( J' \in \cJ' \), s.t. \( x\in J' \implies \lambda x \in J' \subseteq J \).

	See that \( J \ne \emptyset \) since \( 0 \in J \). So the \textbf{conclusion} is \( J \) is an ideal. And it is clear that \( I \subseteq J \).

	See that \( J \ne R \), otherwise \( 1\in J \implies 1 \in J' \) for some \( J' \in \cJ' \), contradict to the fact that \( J' \ne R \).

	It is clear that \( J' \leq J \; \forall \; J' \in \cJ' \implies J \) is the upperbound for \( \cJ' \). The Apply \textbf{Zorn's Lemma} \ref{lem:zorn}: there exists \( M \in \cJ \) to be the maximal element, and this is a maximal ideal that containing \( I \).
\end{proof}
Similar technique can be used to prove the following proposition, which takes the minimal prime ideal to be:
\[
	\mathfrak p_1 = \bigcap_{\mathfrak p' \in \mathfrak P }\mathfrak p'
\]
where in the following, \( \mathfrak p_0 \) is a fixed prime ideal in \( R \). Notably,
\( \mathfrak p_1 \) is a prime ideal, only when \( \mathfrak P \subseteq \mathfrak p \) is arbitrary
totally ordered set w.r.t. \textcolor{blue}{reverse set inclusion}.
\[
	\mathfrak p = \{\mathfrak q \subseteq \mathfrak p_0 \; | \; \mathfrak q \text{ is a prime ideal in
	} R\}
\]

\begin{proposition}
	Let \( R \) be a commutative ring, every prime ideal \( \mathfrak p \) in \( R \) contains a minimal prime ideal, that is, a prime ideal \( \mathfrak q \) such that there is no prime ideal \( \mathfrak q' \), with \( \mathfrak q' \subsetneq \mathfrak q \).
\end{proposition}

\begin{corollary}
	\label{cor:maximal-exists}
	If \( R\ne \{0\} \) is a commutative ring, then there exists a maximal ideals in \( R \), in particular, there exists a prime ideal.
\end{corollary}

\begin{proof}
	Apply the theorem with \( I = \{0\} \) shows the existence of maximal ideal.
\end{proof}

\begin{proposition}
	Let \( R \) be a commutative ring, and \( \mathfrak p_1, \mathfrak p_2 \) be two distinct maximal ideal, then:
	\[
		\mathfrak p_1 + \mathfrak p_2 = R
	\]
\end{proposition}

\begin{proof}[\textbf{Sketch}]
	It is straightforward once realized that \( \mathfrak p_1 + \mathfrak p_2 \) is actually an ideal by definition.
\end{proof}
Notably for \( \ZZ \) there is \textcolor{blue}{no difference} between the notion of maximal ideal
and prime ideal except for \( \{0\} \) being a prime ideal but not a maximal ideal, \( (p) \) is
always maixmal. Consider \( (p_1, p_2) \) by Bezout's Theorem one shall notice that
\( 1 \in (p_1, a) \) and they must be coprime if distinct and thus directly be \( \ZZ \).

\section{Local Ring}

In this section we shall discuss local rings. \todo{leave some overview!}

\begin{definition}[\textbf{Local Ring}]
	A commutative ring \( R \) is a \underline{local ring} if \( R \) has a \textbf{unique} maximal ideal.
\end{definition}

\begin{proposition}
	For a commutative ring \( R \), the following are equivalent:
	\begin{enumerate}
		\item \( R \) is a local ring. (with maximal ideal \( M = \{a \in R \; | \; a \text{ is not invertible}\} \))
		\item \( R \ne \{0\} \) and for all \( a,b \in R \), s.t. \( a+b = 1 \), either \( a \) or \( b \) is invertible.
	\end{enumerate}
\end{proposition}

\begin{proof}
	Suppose that \( R \) is a local ring with maximal ideal \( M \), then \( M \subseteq \{a \in R \; | \; a \text{ is not invertible}\} \) since \( M \ne R \). If \( a\in R \) is not invertible, then \( (a) \ne R \), by \textbf{Theorem} \ref{thm:maximal}, it is contained in a maximal ideal \( \implies (a) \subseteq M \implies a \in M \).

	In this case, \( R \ne \{0\} \) since \( M \ne R \). If \( a+b = 1 \), since \( 1 \not\in M \) and \( M \) being a subgroup, then either \( a\not\in M \) or \( b \not \in M \implies a\) is invertible or \( b \) is invertible.

	Define \( M = \{a \in R \; | \; a \ne \text{ invertible.}\} \). We claim that \( M \) is an ideal:
	\begin{itemize}
		\item \( 0\in M \) since \( R \ne \{0\} \).
		\item If \( a \in M, \lambda \in R \implies \lambda a \in M \), otherwise \( \exists \; \mu \in R \), s.t. \( (\mu\lambda)a = 1\implies a \) is invertible, leading to contradiction $\lightning$.
		\item If \( a,b\in M \implies c:= a+b \in M \), otherwise, If \( c \) is invertible, then:
		      \[
			      (a+b)c^{-1} = ac^{-1} + bc^{-1} = 1
		      \]
		      then this implies that \( ac^{-1} \) or \( bc^{-1} \) is invertible, then \( a = (ac^{-1})c \) is also invertible, similar for \( b \) is invertible, leading to contradiction \( \lightning \).
	\end{itemize}

	So we see that \( M \) is an ideal, remains to check that it is the only maximal ideal.
	\begin{itemize}
		\item \( 1\not\in M \implies M \ne R \).
		\item If \( I \ne R \) is an ideal, then \( I \subseteq M \): \( I\ne R \implies 1 \not\in I \), thus \( \forall \; a\in I \), see that \( a \) is not invertible, since \( (a) \subseteq I \ne R \implies 1 \not\in (a) \implies a\in M \).
	\end{itemize}

	Since we know \( R \) has an maximal ideal by \textbf{Corollary} \ref{cor:maximal-exists}, then \( M \) is a maximal ideal, and in fact the unique one. See that any maximal ideal \( M' \subseteq M \implies M' = M \).
\end{proof}
