\section{\( R \)-subalgebras and \( R \)-algebra of Finite Type\todo{Digest below}}

\begin{definition}[\textbf{\( R \)-subalgebra}]
	Let \( R \) be a commutative ring, together with \( R \xrightarrow{\vp} S \) forms a \( R \)-algebra, the \underline{\( R \)-subalgebra of \( S \)} is given by a subring \( S' \subseteq S \), s.t. \( \vp(R) \subseteq S' \). In this case, we have a \( R \)-algebra struture on \( S' \), s.t. \( S'\hookrightarrow S \) is an \( R \)-algebra homomorphism.
\end{definition}

\begin{proposition}
	If \( (S_i)_{i\in I} \) is family of \( R \)-subalgebra of \( S \) then \( \bigcap_{i\in I} S_i \subseteq S \) is an \( R \)-algebra.
\end{proposition}

\begin{definition}
	If \( A \subseteq S \) is any subset, the \underline{\( R \)-subalgebra of \( S \) generated by \( A \)} is
	\[
		R[A] \coloneqq  \bigcap_{S' \subseteq S \text{be } R \text{-subalgebra}, A \subseteq S'} S'
	\]
\end{definition}

\begin{definition}[\textbf{\( R \)-algebra of Finite Type}]
	\( S \) is an \( R \)-algebra of \underline{finite type} if \( \exists \; A \subseteq S \) be finite, s.t. \( R[A]  = S\).
\end{definition}

\begin{eg}[\textbf{Special Case}]
	Let \( S \) be commutative and \( A = \{a_1, \ldots, a_n\}\). By the universal property of polynomial rings, we have an \( R \)-algebra homomorphism
	\[
		\begin{aligned}
			R[x_1, \ldots, x_n] & \xrightarrow{f} S \\
			f(x_i)              & = a_i
		\end{aligned}
	\]
	\begin{claim}
		\( R[a_1,\ldots, a_n] = \im(f) \)
	\end{claim}
	\begin{proof}
		It's clear that \( A \subseteq \im(f) \subseteq S \) is an \( R \)-subalgebra, thus \( R[a_1,\ldots, a_n] \subseteq \in(f)\). On the other hand, since \( a_i \in R[a_1,\ldots, a_n] \; \forall \; i \), if \( P \in R[x_1,\ldots, x_n] \implies P(a_1,\ldots, a_n) \in R[a_1,\ldots, a_n] \implies \im(f)\subseteq R[a_1,\ldots, a_n]\).
	\end{proof}
\end{eg}

\begin{eg}
	\( \ZZ[\sqrt d] \subseteq \CC \) is the \( \ZZ \)-subalgebra generated by \( \sqrt d \).
	\begin{proof}
		It's clear that the subalgebra generated by \( \sqrt d \subseteq \ZZ[\sqrt d] \) and conversely \( a+b\sqrt d \in  \) subalgebra generated by \( \sqrt d \; \forall \; a,b \in \ZZ\).
	\end{proof}
\end{eg}

\begin{remark}
	If \( R \) is a Noetherian ring and \( S \) is a commutative \( R \)-algebra of finite type, then \( S \) is a Noetherian ring.
	\begin{proof}
		Since \( S \) will be a quotient ring of some \( R[x_1,\ldots, x_n] \) and by Hilbert Basis Theorem \ref{thm:hilbert-basis} this will be Noetherian, and the quotient of a Noetherian ring will still be Noetherian.
	\end{proof}
\end{remark}

\section{Divisibility}

In this section we restrict our view basically on \( R \) being a \textcolor{red}{domain}.

\begin{definition}
	\leavevmode
	\begin{enumerate}
		\item If \( a,b \in R \) with \( b\ne 0 \), say \( b \) \underline{divides} \( a \), or \( b \mid a \) if \( \exists \; c\in R \), s.t. \( a=bc \) (\( \iff (a)\subseteq (b) \)).
		\item If \( a,b \in R \backslash\{0\} \), say \( a \) and \( b \) are \underline{associated} (\( a\sim b \)) if \( a\mid b \) and \( b\mid a \) (\( \iff (a) = (b) \)).
		      \begin{note}
			      This is a \textbf{equivalence relation}, see that \( a\sim b \iff a=ub\) with \( u \) being a unit.
			      \begin{proof}
				      \( \impliedby \) is clear, and for \( \implies \) part, see that \( ac=b, bd=a \) for some \( c,d\in R \implies acd = bd = a\), and since \( R \) being a domain, this means \( cd = 1 \), thus \( d \) is invertible and being a unit.
			      \end{proof}
		      \end{note}
	\end{enumerate}
\end{definition}

\begin{proposition}[\textbf{Easy Property w/o Proof}]
	\leavevmode
	\begin{enumerate}
		\item If \( a\mid b \) and \( b\mid c\implies a\mid c \).
		\item If \( a\mid b \) and \( a\mid c \implies a\mid \alpha b + \beta c \) \( \forall \; \alpha, \beta \in R \).
	\end{enumerate}
\end{proposition}

\begin{definition}[\textbf{Greatest Common Divisor (gcd)}]
	If \( a,b \in R \) are not both \( 0 \), then a \underline{greatest} \underline{common divisor (gcd)} denoted as \( (a,b) \) or \( \gcd(a,b) \) is \( d\in R \), s.t.:
	\begin{enumerate}
		\item \( d\ne 0 \).
		\item \( d\mid a \) and \( d\mid b \).
		\item \( \forall \; d' \), s.t. \( d'\mid a, d'\mid b \implies d'\mid d \).
	\end{enumerate}
\end{definition}

\begin{note}
	\leavevmode
	\begin{enumerate}
		\item Such a gcd \textcolor{red}{might not exists!}
		\item If \( d \) is a gcd for \( a,b \implies d'\) is also a gcd of \( a,b \) if and only if \( d\sim d' \).
	\end{enumerate}
\end{note}

\begin{proposition}
	If \( a,b \) not both zero and \( (a,b) = (d) \implies d = \gcd (a,b)\). In particular, in a \textcolor{red}{PID}, any \( 2 \) element that are not both zero \textbf{have a gcd}.
\end{proposition}

\begin{proof}
	Let \( a,b \in (d) \implies d\mid a\) and \( d\mid b \). \( d\in (a,b) \implies \exists \; \alpha, \beta \in R \), s.t. \( d = a\alpha + b\beta \). If \( d'\mid a, d'\mid b \implies d'\mid d \). Hence \( d \) is a a gcd of \( a,b \).
\end{proof}

\begin{definition}[\textbf{Prime}]
	If \( a\in R \) is \textbf{non-zero}, then \( a \) is \underline{prime} if \( (a) \) is a prime ideal, namely:
	\begin{enumerate}
		\item \( a\ne \) unit.
		\item whenever \( b,c \in R \) are s.t. \( a\mid bc \implies a\mid b \) or \( a\mid c \).
	\end{enumerate}
\end{definition}

\begin{definition}[\textbf{Irreducible}]
	Let \( a\in R \) be \textbf{non-zero}, it is \underline{irreducible} if
	\begin{enumerate}
		\item \( a \ne \) unit.
		\item whenever \( a=bc \) for some \( b,c \in R \), either \( b \) or \( c \) is a unit.
	\end{enumerate}
\end{definition}

\begin{proposition}
	If \( a\in R \) is prime, then \( a \) is irreducible.
\end{proposition}

\begin{proof}
	It's clear that \( a \ne \) unit. If \( a =bc \) for some \( b,c \in R \implies a\mid bc \implies a\mid b\) or \( a\mid c \). Say \( a\mid b\implies b=au \) for some \( u\in R \). Thus \( a=auc \implies uc = 1 \implies \) \( c \) is a unit.
\end{proof}

\begin{remark}
	The converse is \textcolor{red}{false} in general.
\end{remark}

\begin{definition}{\textbf{Unique Factorization Domain (UFD)}}
	A domain \( R \) is a \underline{unique factorization domain} \underline{(UFD)} if:\todo{seems a compilation error}
	\begin{itemize}
		\item \( \forall \; a\in R, a\ne 0 \), \( a \) is not a unit. We can write \( a = p_1 \cdots p_r \) where \( p_1,\ldots, p_r \) are \textbf{irreducible elements}.
		\item This decomposition is essentially \textbf{unique}: If \( a = q_1\cdots q_s \) is another such decomposition, then:
		      \begin{itemize}
			      \item \( r=s \).
			      \item after reordering the \( q_i \), we have \( p_i \sim q_i \; \forall \; i \).
		      \end{itemize}
	\end{itemize}
\end{definition}
