\begin{eg}
  If \( A = \{a\} \), then the left ideal generated by \( A \) is:
  \[
    Ra = \{\lambda a \; | \; \lambda \in R\}
  \] 
\end{eg}

\begin{remark}
  \leavevmode 
  \begin{enumerate}
    \item For any \( A \), we have left/right ideal generated by \( A \) is:
      \[
        \sum_{a\in A} Ra \quad (\text{resp. } \sum_{a\in A} aR)
      \]
    \item If \( A = \{a_1, \ldots, a_n\} \) and \( R \) be a commutative ring, then the ideal generated by \( A \) is denoted as \( (a_1, \ldots, a_n) \), which is:
      \[
        (a_1, \ldots, a_n) \coloneqq \{\lambda_1 a_1+\cdots + \lambda_n a_n \; | \; \lambda_1, \ldots, \lambda_n \in R\}
      \] 
    \item We say \( A \) is a \textbf{system of generators} of \( I \), where \( I \) being a left/right/2-sided ideal, if \( I \) is \underline{such} ideal generated by \( A \).
  \end{enumerate}
\end{remark}

\begin{eg}
  If \( R \) is commutative, a principal ideal in \( R \) is an ideal generated by \( 1 \) element: \( (a), \; a\in R \).
\end{eg}

\subsection{Product of Ideals}

Let \( R \) be commutative ring, \( I_1, \ldots, I_n \subseteq R \) be ideals, then define:
\[
  I_1\cdots I_n = \text{ideal genearted by } \{a_1a_2\cdots a_n \; | \; a_j \in I_j \; \forall \; j\}
\] 

This means that it can be written as:
\[
  I_1\cdots I_n \coloneqq  \{\sum_{k=1}^d a_{k_1}a_{k_2}\cdots a_{k_n} \; | \; d\in \ZZ_{>0}, \; a_{k_j}\in I_j \; \forall \; j\}
\] 

Suppose that \( f: R \to S \) be a ring homomorphism of commutative rings. If \( I\subseteq R \) be an ideal, what can we say about \( f(I) \)?
\begin{itemize}
  \item \( f(I) \subseteq S \) is a subgroup.
  \item \( f(I) \) is not necessarily an ideal, as it is if and only if \( f \) is \textbf{surjective}.
\end{itemize}

Thus we can look into the ideal generated by \( f(I) \), which is denoted by \( IS \) or \( I\cdot S \):
\begin{equation}
  \label{eq:image-gen-ideal}
  IS \coloneqq \{\sum_{j=1}^n a_j f(b_j) \; | \; n \in \ZZ_{>0}, \; a_j \in S, \; b_j \in I\}
\end{equation}

\begin{eg}
  Suppose that \( S = T^{-1}R \) where \( T\subseteq R \) be a multiplicative system. Let \( I \subseteq R \) be an ideal. See that:
  \[
    IS = T^{-1}I = \{\frac{a}{s} \; | \; a\in I, \; s\in T\}
  \] 
\end{eg}

\section{Spectrum of a Commutative Ring}

This connects closely on topology. Basically it allow us to glue several ring to get some geometric shape.

\begin{definition}
  Given a commutative ring \( R \), define:
  \[
    \spec R \coloneqq  \{P \subset R \; | \; P \text{ is prime ideal.}\}
  \] 

  For every ideal (not necessarily prime) \( I \subset R \), let 
  \[
    V(I) \coloneqq  \{P \in \spec R \; | \; I \subseteq P\}
  \] 
\end{definition}

Note that we consider it as some sort of topology with closed set being \( V(I) \) for some \( I \subset R \) being ideal.

\begin{proposition}[\textbf{Zariski Topology}]
  We have a topology on \( \spec R \), s.t. the \textbf{closed sets} are the \( V(I) \) ofr \( I \subset R \). The topology it forms is called \underline{Zariski Topology}.
\end{proposition}

\begin{proof}
  To verify topology property we basically need to check:
  \begin{enumerate}
    \item \( \spec R = V(I) \) for some \( I \).

      This follows by taking \( I = \{0\} \).
    \item \( \emptyset = V(I) \) for some \( I \).

      This follows by taking \( I = R \).
    \item \( \forall \; (I_\alpha) \) be ideals in \( R \), we have
      \[
        \bigcap_{\alpha \in \Lambda} V(I_\alpha) = V(J) \quad \text{ for some } J 
      \] 

      Consider let \( P \in \bigcap_{\alpha}V(I_\alpha) \iff P \supseteq I_{\alpha} \; \forall \; \alpha \iff P\subseteq \sum_{\alpha \in \Lambda}I_\alpha\). So this follows by taking \( J = \sum_{\alpha} I_\alpha \).
    \item \( \forall \) ideals \( I_1, I_2 \subseteq R \), we have:
      \[
        V(I_1) \cup V(I_2) = V(J) \quad \text{ for some } J 
      \] 

      We try to show that \( V(I_1) \cup V(I_2) = V(I_1\cap I_2) \), the ``\( \subseteq \)'' part is clear simply by the fact that \( I_1\cap I_2 \subseteq I_1 \) and \( I_1\cap I_2 \subseteq I_2 \). Now suppose that \( P \in V(I_1\cap I_2) \), then see that \( I_1\cap I_2 \subseteq P \). If \( I_1\not\subseteq P, I_2 \not\subseteq P \implies \exists \; x_1\in I_1 - P, \; x_2 \in I_2 - P \), s.t. \( x_1x_2 \in I_1 \cap I_2 \) (follows by ideal property) but \( x_1 x_2 \not\in P \) since \( P \) is prime ideal, and reason by contraposition. This leads to contradiction $\lightning$. Hence by contradiction, either \( P \supseteq I_1 \) or \( P \supseteq I_2 \), thus \( P \in V(I_1) \cup V(I_2) \).
  \end{enumerate}
\end{proof}

\begin{eg}
  Let \( R = \ZZ \implies \) the prime ideals are precisely \( \{\{0\}, p\ZZ \text{ for } p \text{ prime.}\} \)\todo{see next time every ideal is principal in this case}. Then:
  \begin{itemize}
    \item \( \{p\ZZ\} \) are closed.
    \item \( \overline{\{(0)\}} = \overline{\{0\}} = \spec(\ZZ) \).
  \end{itemize}
\end{eg}
In fact, we can define a \textbf{functor} as follows:
\[
  \underline{\text{CommutativeRings}} \longrightarrow \underline{\Top^\circ}
\] 
where LHS is the category of commutative rings, and the RHS is the dual of the category of topological spaces.

Recall the definition of a functor:

\begin{definition}[\textbf{Functor}]
  If \( \cC \) and \( \cD \) are categories, then a functor:
  \[
    F: \cC \to \cD 
  \] 

  is given by:
  \begin{enumerate}
    \item For every \( X \in \ob(\cC) \), we have \( F(X) \in \ob(\cD) \).
    \item For all \( X, Y \in \ob(\cC) \), we have a map 
      \[
        \Hom_\cC(X,Y) \to \Hom_\cD(F(X), F(Y))
      \] 

      s.t.
      \begin{enumerate}
        \item \( F(1_X) = 1_{F(X)} \quad \forall \; X \in \ob(\cC)\).
        \item \( \forall \; u \in \Hom_\cC(X,Y), v \in \Hom_\cC(Y,Z) \), preserving the composition structures of the morphisms.
          \[
            F(u\circ v) = F(v) \circ F(u)
          \] 
      \end{enumerate}
  \end{enumerate}
\end{definition}


If \( f: R \to S \supseteq P \) to be a ring homomorphism between commutative rings, one can define:
\[
  \begin{aligned}
    f^\# : \spec(S) & \to \spec(R) \\ 
    f^\# (P) &= f^{-1} (P)
  \end{aligned}
\] 

We know that \( f^{-1}(P) \subseteq R \) is an ideal, it's ``inverse'' is ``surjective'', notably to see that it is actually a \textbf{prime ideal}:
\[
  \text{If } xy \in f^{-1}(P) \implies f(xy) = f(x) f(y) \in P \implies f(x) \in P \text{ or } f(y) \in P \implies x\in f^{-1}(P) \text{ or } y \in f^{-1}(P)
\] 

\begin{claim}
  \( f^\# \) is \textbf{continuous}.
\end{claim}

\begin{proof}
  It is enough to show that \( (f^\#)^{-1} \)(closed set) is also closed.

  Fix \( I \subseteq R \) be an ideal, then:
  \[
    (f^\#)^{-1} (V(I)) = \{P \subseteq S \; | \; P \text{ prime. } f^\#(P)\supseteq I \iff f^{-1}(P) \supseteq I \iff P \supseteq f(I) \iff P \supseteq IS \;\; \ref{eq:image-gen-ideal}\}
  \] 

  Thus we can conclude: \( (f^\#)^{-1}(V(I)) = V(IS) \) which is closed.
\end{proof}
To check that it is a functor, need:
\begin{enumerate}
  \item \( (\Id_R)^\# = \Id_{\spec R} \), which is clear.
  \item The following is true since \( (g\circ f)^{-1} (P) = f^{-1}(g^{-1}(P)) \):
    \[
      \begin{aligned}
        R \xrightarrow{f} S & \xrightarrow{g} T \\ 
        (g\circ f)^\# =& f^\# \circ g^\# 
      \end{aligned}
    \] 
\end{enumerate}
We can actually also do similar things for maximal ideal!

\begin{remark}
  May consider:
  \[
    \maxspec(R) \coloneqq  \{P \subset R \; | \; P \text{ maximal ideal.}\} \subseteq \spec(R) \text{ w.r.t. subspace topology.}
  \] 
  \begin{itemize}
    \item It \textbf{doesn't} give a functor in general.
    \item It indeed \textbf{well-behaved} and useful in geometry if \( R \) is a quotient of a polynomial ring, namely: \( \qo{\KK[x_1, \ldots,x_n]}{I} \) where \( \KK \) be a field, and in this case, it indeed \textbf{gives a functor}.
  \end{itemize}
\end{remark}
