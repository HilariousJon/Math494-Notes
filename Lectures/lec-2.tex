\section{Subrings and Ideals}

We consider the subobjects of ring in this section. In particular, note that sometimes people define
rings by unitary ring, in such case Ideals are exactly normal \textbf{ring}.

\begin{definition}[\textbf{Subring}]
	Let \( R \) be a ring. A \underline{Subring} of \( R \) is a subset \( S \), s.t. \( (+), (\cdot) \) in \( R \) induce operations on \( S \) that make \( S \) a ring with unit \( 1_R \).
\end{definition}

\begin{remark}
	Definition of subrings implies:
	\begin{enumerate}
		\item \( \forall \; a,b \in S \), we have \( a+b \in S \).
		\item \( \forall \; a,b\in S \), we have \( a\cdot b \in S \).
		\item With respect to these operations, \( S \) is a ring with unit \( 1_R \).
	\end{enumerate}
\end{remark}

\begin{proposition}
	If \( R \) is a ring, a subset \( S \subseteq R \) is a subring if and only if:
	\begin{enumerate}
		\item \( a-b \in S, \; \forall \; a,b \in S \).
		\item \( ab\in S, \; \forall \; a,b \in S \).
		\item \( 1_R \in S \).
	\end{enumerate}
\end{proposition}

\begin{proof}
	Only left to proof if 1,2,3 holds, then \( S \) is a ring with unit \( 1_R \) w.r.t. the induced operations.
	\begin{itemize}
		\item \( S \) is a subgroup w.r.t. \( (+) \): By 3, \( S \ne \emptyset \), hence by 1, \( S \) is a subgroup. \( R \) is abelian thus \( S \) is also abelian.
		\item \( 1_R \in S \), this is the identity w.r.t. also in \( S \).
		\item Associativity of \( (\cdot) \) and distributivity also holds in \( S \) because they hold in \( R \).
	\end{itemize}
\end{proof}

\begin{eg}
	\leavevmode
	\begin{enumerate}
		\item \( \ZZ \subseteq \QQ, \QQ \subseteq \RR, \RR \subseteq \CC \) are all subrings.
		\item \( \{\text{even numbers}\} \subseteq \ZZ \) is not a subring since it doesn't contain \( 1 \).
	\end{enumerate}
\end{eg}

\begin{proposition}
	If \( f: R\to S \) is ring homomorphism, then \( \im(f) \subseteq S \) is a subring.
\end{proposition}

The proof is straightforward. With side note that \( f(1_R) = 1_S \in \im(f) \).

\begin{definition}[\textbf{Ideal}]
	Suppose \( R \) be a ring and \( I \subseteq R \) and \( I \ne \emptyset \). Then
	\begin{enumerate}
		\item \( I \) is a \underline{left ideal} (preseve multiplication on the \textbf{left}) if:
		      \begin{itemize}
			      \item \( a+b \in I \; \forall \; a,b \in I \).
			      \item \( \forall \; a\in R, b \in I \implies ab \in I \).
		      \end{itemize}
		\item \( I \) is a \underline{right ideal} (preserve multiplication on the \textbf{right}) if:
		      \begin{itemize}
			      \item \( a+b \in I \; \forall \; a,b \in I \).
			      \item \( \forall \; a \in I, b\in R \implies ab \in I\).
		      \end{itemize}
		\item \( I \) is a \underline{two-sided ideal} if it is both left and right ideal.
	\end{enumerate}

	If \( R \) is \textbf{commutative}, then all the above definition coincide, so we simply say \underline{ideal} in this case.
\end{definition}

\begin{remark}
	\leavevmode
	\begin{enumerate}
		\item Every (left/right) ideal is a \textbf{subgroup}.
		      \begin{itemize}
			      \item \( I \ne \emptyset \implies \exists \; a\in I \implies 0a = 0 \in I \).
			      \item \( \forall \; a\in I \implies -a \in I, \; -a = (-1)a = a \cdot (-1) \).
		      \end{itemize}
		\item If \( I \) is a left (or right) ideal and \( 1 \in I \), then \( I = R \), since \( \forall a \in R, \; a = a \cdot 1 \in I \).

		      Hence the only subring that is a left or right ideal is \( R \).
		\item The \textcolor{red}{addition axiom} of ideal is w.r.t. \textbf{finite sum}, when there comes to infinite sum, things can be different. This is quite like the finite intersection axiom for the definition of general topology.
		\item When one wants to approach to the idea that some ideal \( I \) equals to the whole ring \( R \), try to deduce that \textcolor{red}{\( 1_R \in I \)}, or say the ideal \( I \) contains a \textcolor{red}{unit}.
	\end{enumerate}
\end{remark}

\begin{eg}
	\leavevmode
	\begin{enumerate}
		\item \( R \) and \( \{0\} \) are always two-sided ideals in \( R \).
		\item Say \( R \) is \textbf{commutative} and \( a \in R \), let \( (a) \) to be the subset of \( R \) which contain all multiples of \( a \):
		      \[
			      (a) \coloneqq \{ab \; | \; b \in R\}
		      \]

		      is an ideal in \( R \), such ideal are called \textbf{Principal Ideals}.
		      \begin{itemize}
			      \item \( (a) \ne \emptyset \) since \( a = a \cdot 1 \in (a) \).
			      \item \( ab_1 + ab_2 = a(b_1 + b_2) \in (a) \).
			      \item \( c\in R \), \( (ab)c = a(bc) \in (a) \)
		      \end{itemize}
	\end{enumerate}
\end{eg}

\begin{proposition}
	If \( f: R \to S \) is a ring homomorphism, then \( \ker(f) := \{a\in R \; | \; f(a) = 0 \} \) is a two-sided ideal of \( R \).
\end{proposition}

\begin{proof}
	We know it is a subgroup of \( (R, +) \), see that:
	\[
		a\in \ker(f) \implies f(ba) = f(b) \cdot f(a) = f(b) \cdot 0 = 0 \implies ba \in \ker(f)
	\]

	similarly, \( ab\in \ker(f) \; \forall \; b \in R \).
\end{proof}

Also note, \( f: R\to S \) be ring homomorphism, it is injective iff \( \ker(f) = \{0\} \).

\begin{proposition}
	Let \( R_1, R_2 \) be a commutative ring, and let \( R = R_1 \times R_2 \). Then the ideal of \( R \) will take form of:
	\[
		I_1 \times I_2
	\]
	where \( I_1 \) and \( I_2 \) be some ideal of \( R_1, R_2 \) respectively. Moreover, if \( P \) be prime ideal of \( R \), then it will take form in:
	\begin{itemize}
		\item \( R_1 \times P_2 \) for some \( P_2 \) be prime ideal in \( R_2 \).
		\item \( P_1 \times R_2 \) for some \( P_1 \) be prime ideal in \( R_1 \).
	\end{itemize}
\end{proposition}

\begin{proof}[\textbf{Sketch}]
	Consider the projection map \( \pi \) as the group homomorphism, and see that the \textbf{images}
	are the ideal. As for the prime ideal case, write out the definition and one should observe that
	there are only two case for a prime ideal to take form:
	\begin{itemize}
		\item \( P_1 \times P_2 \), actually not a prime ideal, one can see by verify the definition.
		\item \( R_1 \times R_2 \), not the prime ideal since it is the whole ring.
		\item \( R_1 \times P_2 \), ok.
		\item \( P_1 \times R_2 \), ok.
	\end{itemize}
\end{proof}
