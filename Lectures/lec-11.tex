\section{The Rings \( \ZZ[\sqrt{d}] \) and \( \QQ[\sqrt{d}] \)}

\begin{definition}
	Fix \( \mathbf{d\in \ZZ} \), with \( \abs{d} \) not a \underline{square}: namely \( \not\exists \; n \in \ZZ \), s.t. \( n^2 = d \). This implies that \( \sqrt{\abs{d}} \not\in\QQ \). We write \( \sqrt{\abs{d}} = \sqrt{-d} \) if \( d< 0 \). Thus define:
	\[
		\begin{aligned}
			\ZZ[\sqrt{d}] & = \{a+ b \sqrt{d} \; | \; a,b \in \ZZ\} \subseteq \CC \\
			\QQ[\sqrt{d}] & = \{a + b\sqrt{d} \; | \; a,b \in \QQ\} \subseteq \CC
		\end{aligned}
	\]
\end{definition}

\begin{remark}
	\leavevmode
	\begin{enumerate}
		\item Since \( \sqrt{d} \not\in \QQ \), thus if \( a+b\sqrt{d} = a'+b'\sqrt{d}\) for some \( a,a',b,b' \in \QQ \implies a=a', b=b' \).
		\item \( \ZZ[\sqrt{d}] \) and \( \QQ[\sqrt{d}] \) are subring of \( \CC \), thus they are \textbf{domains}.
		\item \( \QQ[\sqrt{d}] \) is a field, as one can multiply by its conjugate. If \( u = a+b\sqrt{d} \implies u \cdot (a-b\sqrt{d}) = a^2 -b^2 d \in \QQ \backslash \{0\} \), thus:
		      \[
			      u^{-1} = \frac{a}{a^2 - db^2} - \frac{b}{a^2-db^2} \sqrt{d}
		      \]
		\item Since \( \ZZ[\sqrt{d}] \subseteq \QQ[\sqrt{d}] \), by the universal property of fraction field \( \KK \) of \( \ZZ[\sqrt{d}] \) gives us: \(f: \KK \to \QQ[\sqrt{d}] \) as a ring homomorphism.
		      \begin{itemize}
			      \item This is injective since \( \KK \) is a field, since clearly the map is induced by \( \ZZ[\sqrt{d}] \subseteq \QQ[\sqrt{d}]\) so it is not a zero map. The field has no non-trivial ideals and \( \ker(f) \), thus the only case would be \( \ker(f) = \{0\} \), yields injectivity.
			      \item This is surjective as for \( a+b\sqrt{d} \in \QQ[\sqrt{d}] \), one can always write:
			            \[
				            a+b\sqrt{d} = \frac{p_1q_2+p_2q_1\sqrt{d}}{q_1q_2} \text{ with }p_1 q_2 +p_2q_1 \sqrt{d} \text{ and }q_1 q_2 \in \ZZ[\sqrt{d}]
			            \]
			            Thus yields to be \textbf{isomorphism}. Note here the multiplicative system yields to be \( \ZZ[\sqrt{d}] \backslash\{0\} \) which is the only case can be constructed as field for \( \KK \).
		      \end{itemize}
		      In particular this tells us \textbf{up to isomorphism}, the fraction field of \( \ZZ[\sqrt{d}] \) is \( \QQ[\sqrt{d}] \).
		\item Define:
		      \[
			      \begin{aligned}
				      \ZZ[\sqrt{d}]    & \xrightarrow{\vp} \ZZ[\sqrt{d}]
				      \vp(a+b\sqrt{d}) & = a - b\sqrt{d}
			      \end{aligned}
		      \]
		      when \( d \) is negative, this is exactly the conjugation for complex number. See that \( \vp \) is actually a \textbf{ring homomorphism} and in particular \( \vp \circ \vp = \Id \implies \vp \) is actually a ring isomorphism.
		\item Define:
		      \[
			      \begin{aligned}
				      N : \ZZ[\sqrt{d}] & \to \ZZ[\sqrt{d}]                                  \\
				      N(u)              & = u \cdot \vp(u)                                   \\
				      N(a+b\sqrt{d})    & = (a+b\sqrt{d}) (a-b\sqrt{d}) = a^2 - db^2 \in \ZZ
			      \end{aligned}
		      \]
		      \begin{note}
			      \leavevmode
			      \[
				      N(uv) = N(u) N(v) \quad \forall \; u,v \qquad \text{\hypertarget{eq:hom}{($\star\star$)}}
			      \]
			      since \( \vp(uv) = \vp(u)\vp(v) \). But similar result doesn't holds for addition.
		      \end{note}
	\end{enumerate}
	\begin{eg}
		\( u = a+b\sqrt{d} \) is invertible in \( \ZZ[\sqrt{d}] \) if and only if \( N(u) = \pm 1 \).
		\begin{proof}
			If \( u \) is invertible, then \( uv = 1 \) for some \( v \in \ZZ[\sqrt{d}] \), thus:
			\[
				N(u)N(v) = N(uv) = N(1) = 1
			\]
			Since \( N(u), N(v) \) are integers, this implies that \( N(u) = \pm 1 \).

			Conversely, if \( u\cdot \vp(u) = N(u) = \pm 1 \implies u^{-1} = \pm \vp(u) \).
		\end{proof}
	\end{eg}
\end{remark}

\begin{eg}[\textbf{Gauss Integers: \( d = -1 \)}]
	The \underline{Gauss Integers} is given as:
	\[
		\ZZ[i] \coloneqq  \{a+bi \; | \; a,b \in \ZZ\}
	\]
	and \( N(a+bi) = a^2 + b^2 \). We see \( a+bi \) is a \underline{unit} (invertible elments in a ring) if and only if \( a^2 + b^2 = 1 \implies \) the units are exactly \( \pm 1, \pm i \).
\end{eg}
