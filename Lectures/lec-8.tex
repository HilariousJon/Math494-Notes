\begin{eg}
	\leavevmode
	\begin{enumerate}
		\item \( \KK \) is a field \( \implies \KK \) is a local ring.
		\item Let \( R \) be a commutative ring, \( P\subseteq R \) be a prime ideal, define \( S = R - P \), and thus be a multiplicative system. Define \( R_p := S^{-1} R \). By HW \#3:
		      \[
			      \{\text{Prime ideal in } S^{-1}R\} \xleftrightarrow{\text{order preserving bij.}} \{\text{Prime ideal }q \text{ in } R \text{ with } S\cap q = \emptyset \iff q \subseteq P\}
		      \]
		      where order preserving means the bijection is compatible with inclusion. This implies that \( S^{-1} R \) is a local ring with maximal ideal:
		      \[
			      S^{-1}P = \{\frac{a}{s} \in R_p \; | \; a\in P\}
		      \]
		      \begin{eg}
			      If \( p\in \ZZ_{>0} \) is a prime integer, then:
			      \[
				      \ZZ_{(p)} = \{\frac{a}{b} \; | \; a,b \in \ZZ, \; (p\nmid b) \iff (S = \ZZ - (p))\}
			      \]
			      with the maximal ideal being:
			      \[
				      \{\frac{pa}{b} \; | \; a,b \in \ZZ, \; p\nmid b\} \quad \text{by } pa \in (p)
			      \]
		      \end{eg}
	\end{enumerate}
\end{eg}

The point is that if we want to study the property of the ring, we can sometimes go to the local ring and use their properties.

\section{Radical Ideals}

In this section we shall discuss radical rings.\todo{leave some overview!}

\begin{definition}[\textbf{Radical Ideal}]
	Let \( R \) be commutative ring, and \( I\subseteq R \) be an ideal. \( I \) is a \underline{radical ideal} if \( \forall \; a\in R \), s.t. \( a^n \in I \) for some \( n\geq 1 \implies a\in I \).
\end{definition}

\begin{definition}[\textbf{Reduced Ring}]
	Let \( R \) commutative ring, \( R \) is called a \underline{reduced ring} if \( \{0\} \) is a radical ideal.
\end{definition}

\begin{theorem}
	Let \( R \) be commutative ring, and \( I\subseteq R \) be an ideal. \( I \) is a radical ideal if and only if \( \qo{R}{I} \) is a reduced ring.
\end{theorem}

The proof is straightforward by just formally write out the definition.

\begin{eg}
	\leavevmode
	\begin{enumerate}
		\item \( I \) is a prime ideal \( \implies I\) is a radical ideal. As one can consider \( a^n \in I \implies (a^{n-1})a\in I \implies \) either \( a^{n-1}\in I \) or \( a\in I \) and doing this repeatedly eventually leads to \( a\in I \).
		\item If \( n\in \ZZ_{>0} \), then \( (n) \) is radical ideal if and only if \( n \) is square free, namely if \( n = p_1^{a_1} \cdots p_r^{a_r} \) to be the prime decomposition, then \( a_i = 1 \; \forall \; i \).
		      \begin{proof}[\textbf{Sketch of Proof}]
			      One can consider the prime factorization of \( n \) as:
			      \[
				      n = p_{i_1}^{a_{i_1}} \cdots p_{i_r}^{a_{i_r}}
			      \]
			      and consider arbitrary element \( a\in \ZZ \) such that \( a^k \in (n) \), the prime factorization of \( a^k \) given by:
			      \[
				      a^k = p_{j_1}^{b_{j_1}}\cdots p_{j_l}^{b_{j_l}} = c n \qquad \text{for some } c \in \ZZ
			      \]
			      and since \( c \) is integer it means that LHS should cancel out all the prime factors of \( n \), in particular, this means that:
			      \[
				      \{i_1, \ldots, i_r\} \subseteq \{j_1, \ldots, j_l\}
			      \]
			      and cancel things out one can still write \( a = dn \) for some \( d\in \ZZ \).
		      \end{proof}
	\end{enumerate}
\end{eg}

\begin{remark}
	\leavevmode
	\begin{enumerate}
		\item We showed in HW\#2, that:
		      \[
			      \rad(I) = \{a\in R \; | \; a^n \in I \text{ for some } n\geq 1\}
		      \]
		      is an ideal in \( R \). See that \( I\subseteq \rad(I) \), with equality if and only if \( I \) is a \textbf{radical ideal}.
		      \begin{proof}[\textbf{Sketch of Proof}]
			      Its straightforward to see that \( I \subseteq \rad(I) \). When \( I \) is a radical ideal, this means that \( a^n \in I \implies a\in I \implies \rad(I) \subseteq I \implies I = \rad(I)\). When \( I = \rad(I) \), then \( \rad(I) \subseteq I \implies a^n \in I \implies a \in I \).
		      \end{proof}
		\item \( \rad(I) \) is a radical ideal. Just check above proof.
		\item \( \rad(I) \) is the \textbf{smallest} radical ideal containing \( I \).
		\item \leavevmode
		      \begin{parenthesis}
			      If \( (I_\alpha)_\alpha \) is a family of left/right/two-sided ideals in \textbf{any ring} \( R \), then:
			      \[
				      \bigcap_{\alpha}I_\alpha \text{ has the same property.}
			      \]
		      \end{parenthesis}
		\item If each \( I_\alpha \) is a radical ideal, then \( \bigcap_{\alpha}I_\alpha \) is also a radical ideal. The proof is quite straightforward.
		      \begin{note}
			      This is \textbf{false} for prime ideals, see that in \( \ZZ \), \( (2) \cap (3) = (6) \), where \( (6) \) is not a prime ideal.

			      However, if the family is a \textbf{totally ordered set characterized by set inclusion}, then this statement is true. See homework related to Zorn's Lemma, this also gives us the statement that \textbf{every prime ideals have a minimal prime ideal}.
		      \end{note}
	\end{enumerate}
\end{remark}

\begin{proposition}
	For every ideal \( I \subseteq R \), see that:
	\[
		\rad(I) = \bigcap_{P\supseteq I, P \text{ prime ideal.}} P
	\]
\end{proposition}

\begin{proof}
	First note that:
	\[
		I\subseteq \underbrace{\bigcap_{I\subseteq P, P \text{ prime ideal}} P}_{\text{radical, since prime is radical and intersection of radical is radical}} \implies \rad(I) \subseteq \bigcap_{I\subseteq P, P \text{ prime ideal}} P
	\]
	since \( \rad(I) \) is the smallest radical ideal that contains \( I \).

	Suppose \( f\in \bigcap_{I\subseteq P, P \text{ prime ideal}} P \), we want to see that \( f^n \in I \) for some \( n\geq 1 \).

	\textcolor{red}{The general ideal here is to replace \( (R, I, f) \) by \( (\qo{R}{I}, \{0\}, \overline{f}) \)}, as one can see that:
	\[
		\overline{f}^n = 0 \iff f^n \in I
	\]
	big picture is that the property \( f^n \in I \) is \textbf{carried} by ring homomorphism, and one will make it easier to consider in quotient ring, and further to fraction it out using the multiplicative system \( S = \{1,f,f^2,\ldots\} \).

	May assume \( I = \{0\} \), thus \( f\in P \) for all prime ideal \( P \). The \textbf{tricks} here is to consider \( R_f = S^{-1}R \) where \( S \) is defined as above. The prime ideals in \( R_f \) is the same as the prime ideals \( P \) in \( R \), s.t. \( S\cap P = \emptyset \iff f\not\in P \). And see that there are no such prime ideals in \( R \), and so there will be no such prime ideal in \( S^{-1}R \), but we've seen in \textbf{Theorem} \ref{thm:maximal} that every commutative ring who have a proper ideal should have a maximal ideal, and maximal ideal is prime ideal, and itself is an ideal, it follows that \( R_f = \{0\} \). So:
	\[
		R_f = \{0\} \implies \frac{0}{1} = \frac{1}{1} \iff \exists\; n, \; s.t. \; f^n = 0
	\]
\end{proof}

\begin{corollary}
	An ideal is a radical ideal if and only if it is the intersection of all prime ideals who contains it.
\end{corollary}

\begin{proof}
	As prime ideals are radical ideal, the forward direction trivially holds. The reverse direction directly yields combining the proposition and the fact that \( I \) is radical if and only if \( I = \rad(I) \).
\end{proof}

\section{Operations with Ideals}

In this section we will see several operators to help us construct more and more ideals from existing ideals.

\subsection{Sum of Ideals}

Let \( R \) be any ring, then we've seen that the intersection of ideals are ideals, we now define the \underline{sum of ideals} for \( (I_\alpha)_{\alpha \in \Lambda} \).

Let \( I_\alpha \) be left/right/2-sided ideal in \( R \), define the sum of them as:
\[
	\sum_{\alpha \in \Lambda} I_\alpha \coloneqq \bigcap_{I \text{ be \underline{such}\todo{such here means corresponding left/right/2-sided} ideal } I_\alpha \subseteq I \; \forall \; \alpha} I
\]
This is the unique \textbf{smallest} ideal containing all \( I_\alpha \). Note that we consider \textbf{finite sum} here, if it is infinite sum, then we put finitely of them that are non-zero.

\begin{proposition}[\textbf{Equivalence def. of Sum of Ideals}]
	\leavevmode
	\[
		\sum_{\alpha \in \Lambda}I_\alpha = \{\sum_{\alpha \in \Lambda} a_\alpha \; | \; a_\alpha \in I_\alpha \; \forall \; \alpha, \text{ only finitely many } a_\alpha \text{ are } \ne 0\}
	\]
	\begin{eg}
		\leavevmode
		\[
			I_1 + I_2 = \{a+b \; | \; a \in I_1, \; b \in I_2\}
		\]
	\end{eg}
\end{proposition}

\begin{proof}[\textbf{Sketch of Proof}]
	It is straightforward to verify that the RHS is an ideal, and it contains all \( I_\alpha \), thus ``\( \subseteq \)'' part directly yields.

	On the other hand, if \( I \) is an ideal, s.t. \( I_\alpha \subseteq I \; \forall \; \alpha \), then  RHS \( \subseteq I \), which then yields ``\( \supseteq \)'' part.
\end{proof}
More generally, given any subset \( A \subseteq R \), may consider the smallest left/right/2-sided ideal \textbf{generated} by \( A \):
\[
	\bigcap_{I \text{ be \underline{such} ideal } A \subseteq I} I
\]
If \( R \) is commutative, write \( (A) \) for this ideal.
