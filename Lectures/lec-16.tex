\section{Introduction to Algebraic Sets}

In this section we shall give an introduction on algebraic geometry, basically try to build the
bridge between algebra and geometry by showing the famous Hilbert's Nullstellensatz.

Let's fix a algebraic closed field \( k \) (e.g. \( \CC \)) and define \( \bA_{k}^n = k^n\). Define:
\[
	R = k[x_1,\ldots, x_n]
\]
Now given fixed \( a = (a_1,\ldots, a_n) \in \bA^n \), we look into the surjective ring homomprhism:
\[
	\begin{aligned}
		\vp: R & \longrightarrow k                     \\
		f      & \longmapsto f(a_1,\ldots, a_n) = f(a)
	\end{aligned}
\]
The map is simply given by the polynomial evaluation at the point \( a \). We can easily check that
\( \vp \) is a ring homomorphism and it is surjective since \( k \) is algebraically closed. Now we
want to understand the kernel of \( \vp \).
\begin{claim}
	\( \ker(\vp) = (x_1 - a_1, \ldots, x_n - a_n) \)
\end{claim}

\begin{proof}
	For every \( f \in R \), apply the division algorithm for \( f \) w.r.t. \( x_1-a_1, \ldots, x_n -
	a_n\), then for \( f\in R \) we can write:
	\[
		f = (x_1 - a_1)g_1 + \cdots + (x_n - a_n)g_n + b \quad b \in k
	\]
	which implies \( b = f(a) \). Hence \( f(a) = 0 \implies f \in (x_1 - a_1, \ldots, x_n - a_n) \).
	This shows that \( (x_1-a_1,\ldots, x_n - a_n) \) is a maximal ideal since \( k \) being a field
	and implied by first isomorphism theorem.
\end{proof}
Our goal then is to show that every maximal ideal is \textbf{of this form}. We shall define several
useful notions first.

\begin{definition}
	For any ideal \( I \subseteq R \) be ideal, define
	\[
		V(I) \coloneqq  \{a \in \bA^n \; | \; f(a) = 0 \; \; \forall \; f \in I\}
	\]
	Namely those Nullstellen that shares among all the polynomials in the ideal.
\end{definition}

\begin{remark}
	\leavevmode
	\begin{enumerate}
		\item If \( I = (f_1,\ldots, f_r)\implies V(I) = \{a \; | \; f_i (a) = 0 \; \; \forall \; 1 \leq
		      i \leq r\} \).
		      \begin{proof}
			      By \( f = \sum f_i g_i \) and \( f_i (a) = 0 \; \forall \; i \implies f(a) = 0 \).
		      \end{proof}
		\item Since \( R \) is Noetherian, then every \( V(I) \) can be defined by finitely many
		      equations. \( R \) is Noetherian is by the fact that \( k \) is Noetherian, as it only has
		      two ideal, which follows definition of Noetherian. And that by the first remark, finitely
		      generated ideal \( I \) will have \( V(I) \) being expressed by finitely many equations.
	\end{enumerate}
\end{remark}
See that it shall have the following \textbf{Properties}.

\begin{remark}
	\leavevmode
	\begin{enumerate}
		\item \( I \subseteq J \implies V(J) \subseteq V(I) \). As clearly more polynomial adding in
		      will decrease the number of common Nullstellen.
		\item \( V(R) = \emptyset \). As one can clearly find polynomial that don't have any
		      Nullstellens.
		\item \( V(\{0\}) = \bA^n \). As the zero polynomial attains Nullstellen to be the whole space.
		\item If \( (I_\alpha) \) is a family of ideals, shall have:
		      \[
			      V\bace{\sum_{\alpha} I_\alpha} = \bigcap_\alpha V(I_\alpha)
		      \]
		      Since the polynomial summing together will attains their Nullstellens being the
		      intersections of the Nullstellen of the individual polynomial.
		\item If \( I, J \) are ideals, then:
		      \[
			      V(I) \cup V(J) = V(I \cap J) = V(IJ)
		      \]
		      \begin{proof}
			      One shall clearly see that
			      $$
				      \begin{aligned}
					      IJ                      & \subseteq I\cap J                     \\
					      I \cap J                & \subseteq I                           \\
					      I \cap J                & \subseteq J                           \\
					      \implies V(I) \cup V(J) & \subseteq V(I \cap J) \subseteq V(IJ)
				      \end{aligned}
			      $$
			      we want to see that:
			      \[
				      V(IJ) \subseteq V(I) \cup V(J)
			      \]
			      If this is not the case, there exists \( a\in V(IJ) \), s.t. \( a\not\in V(I), a\not\in
			      V(J)\). It means that there exists \( f \in I, g\in J \), s.t.
			      \[
				      f(a) \ne 0 \quad g(a) \ne 0 \implies fg \in IJ \text{ but } fg(a) = f(a) g(a) \ne 0
				      \text{ as we are in a domain.}
			      \]
			      But this leads to contradiction as \( a \) is taken as the nullstellen of polys in
			      \( IJ \) \( \lightning \), the proof directly finish the whole inclusion chain.
		      \end{proof}
	\end{enumerate}
\end{remark}

\begin{definition}[\textbf{Algebraic Subsets of \( \bA^n \)}]
	The sets of the form \( V(I) \) are the \underline{algebraic subsets} \underline{of \( \bA^n \)}.
\end{definition}

\begin{definition}[\textbf{Zariski Topology}]
	Properties 2, 3, 4, 5 in the above remark shows that \( V(I) \) forms the closed sets for a
	topology \textbf{on} \( \bA^n \), which is the \underline{Zariski Topology}.
\end{definition}
One can also going the opposite direction by giving a sets of \textcolor{red}{``Nullstellens''} \( Z
\subseteq \bA^n \) and ask the ideal who contains polynomials who have exactly those Nullstellens.

\begin{definition}
	Given \( Z \subseteq \bA^n \), can define:

\end{definition}
