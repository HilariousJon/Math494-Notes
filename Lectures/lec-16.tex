\section{Introduction to Algebraic Sets}

In this section we shall give an introduction on algebraic geometry, basically try to build the
bridge between algebra and geometry by showing the famous Hilbert's Nullstellensatz.

Let's fix a algebraic closed field \( k \) (e.g. \( \CC \)) and define \( \bA_{k}^n = k^n\). Define:
\[
	R = k[x_1,\ldots, x_n]
\]
Now given fixed \( a = (a_1,\ldots, a_n) \in \bA^n \), we look into the surjective ring homomprhism:
\[
	\begin{aligned}
		\vp: R & \longrightarrow k                     \\
		f      & \longmapsto f(a_1,\ldots, a_n) = f(a)
	\end{aligned}
\]
The map is simply given by the polynomial evaluation at the point \( a \). We can easily check that
\( \vp \) is a ring homomorphism and it is surjective since \( k \) is algebraically closed. Now we
want to understand the kernel of \( \vp \).
\begin{claim}
	\( \ker(\vp) = (x_1 - a_1, \ldots, x_n - a_n) \)
\end{claim}

\begin{proof}
	For every \( f \in R \), apply the division algorithm for \( f \) w.r.t. \( x_1-a_1, \ldots, x_n -
	a_n\), then for \( f\in R \) we can write:
	\[
		f = (x_1 - a_1)g_1 + \cdots + (x_n - a_n)g_n + b \quad b \in k
	\]
	which implies \( b = f(a) \). Hence \( f(a) = 0 \implies f \in (x_1 - a_1, \ldots, x_n - a_n) \).
	This shows that \( (x_1-a_1,\ldots, x_n - a_n) \) is a maximal ideal since \( k \) being a field
	and implied by first isomorphism theorem.
\end{proof}
Our goal then is to show that every maximal ideal is \textbf{of this form}. We shall define several
useful notions first.

\begin{definition}
	For any ideal \( I \subseteq R \) be ideal, define
	\[
		V(I) \coloneqq  \{a \in \bA^n \; | \; f(a) = 0 \; \; \forall \; f \in I\}
	\]
	Namely those Nullstellen that shares among all the polynomials in the ideal.
\end{definition}

\begin{remark}
	\leavevmode
	\begin{enumerate}
		\item If \( I = (f_1,\ldots, f_r)\implies V(I) = \{a \; | \; f_i (a) = 0 \; \; \forall \; 1 \leq
		      i \leq r\} \).
		      \begin{proof}
			      By \( f = \sum f_i g_i \) and \( f_i (a) = 0 \; \forall \; i \implies f(a) = 0 \).
		      \end{proof}
		\item Since \( R \) is Noetherian, then every \( V(I) \) can be defined by finitely many
		      equations. \( R \) is Noetherian is by the fact that \( k \) is Noetherian, as it only has
		      two ideal, which follows definition of Noetherian. And that by the first remark, finitely
		      generated ideal \( I \) will have \( V(I) \) being expressed by finitely many equations.
	\end{enumerate}
\end{remark}
See that it shall have the following \textbf{Properties}.

\begin{remark}
	\leavevmode
	\begin{enumerate}
		\item \( I \subseteq J \implies V(J) \subseteq V(I) \). As clearly more polynomial adding in
		      will decrease the number of common Nullstellen.
		\item \( V(R) = \emptyset \). As one can clearly find polynomial that don't have any
		      Nullstellens.
		\item \( V(\{0\}) = \bA^n \). As the zero polynomial attains Nullstellen to be the whole space.
		\item If \( (I_\alpha) \) is a family of ideals, shall have:
		      \[
			      V\bace{\sum_{\alpha} I_\alpha} = \bigcap_\alpha V(I_\alpha)
		      \]
		      Since the polynomial summing together will attains their Nullstellens being the
		      intersections of the Nullstellen of the individual polynomial.
		\item If \( I, J \) are ideals, then:
		      \[
			      V(I) \cup V(J) = V(I \cap J) = V(IJ)
		      \]
		      \begin{proof}
			      One shall clearly see that
			      $$
				      \begin{aligned}
					      IJ                      & \subseteq I\cap J                     \\
					      I \cap J                & \subseteq I                           \\
					      I \cap J                & \subseteq J                           \\
					      \implies V(I) \cup V(J) & \subseteq V(I \cap J) \subseteq V(IJ)
				      \end{aligned}
			      $$
			      we want to see that:
			      \[
				      V(IJ) \subseteq V(I) \cup V(J)
			      \]
			      If this is not the case, there exists \( a\in V(IJ) \), s.t. \( a\not\in V(I), a\not\in
			      V(J)\). It means that there exists \( f \in I, g\in J \), s.t.
			      \[
				      f(a) \ne 0 \quad g(a) \ne 0 \implies fg \in IJ \text{ but } fg(a) = f(a) g(a) \ne 0
				      \text{ as we are in a domain.}
			      \]
			      But this leads to contradiction as \( a \) is taken as the nullstellen of polys in
			      \( IJ \) \( \lightning \), the proof directly finish the whole inclusion chain.
		      \end{proof}
	\end{enumerate}
\end{remark}

\begin{definition}[\textbf{Algebraic Subsets of \( \bA^n \)}]
	The sets of the form \( V(I) \) are the \underline{algebraic subsets} \underline{of \( \bA^n \)}.
\end{definition}

\begin{definition}[\textbf{Zariski Topology}]
	Properties 2, 3, 4, 5 in the above remark shows that \( V(I) \) forms the closed sets for a
	topology \textbf{on} \( \bA^n \), which is the \underline{Zariski Topology}.
\end{definition}
One can also going the opposite direction by giving a sets of \textcolor{red}{``Nullstellens''} \( Z
\subseteq \bA^n \) and ask the ideal who contains polynomials who have exactly those Nullstellens.

\begin{definition}
	Given \( Z \subseteq \bA^n \), can define:
	\[
		I(Z) = \{f \in R \; | \; f(a) = 0 \; \; \forall \; a \in Z\}
	\]
	This is an ideal in \( R \), and in fact it is a \textbf{radical ideal}.
\end{definition}
The following easy properteis are straightforward to check:

\begin{remark}
	\leavevmode
	\begin{enumerate}
		\item If \( Z_1 \subseteq Z_2 \subseteq \bA^n  \implies I(Z_2) \subseteq I(Z_1)\). Which is
		      straightforward as more Nullstellens results in smaller sets of corresponding polynomials
		      who have such Nullstellens.
		\item \( I(Z_1 \cup Z_2) = I(Z_1) \cap I(Z_2) \). This is also straightforwards as those polys
		      that have both of the sets as nullstallens will be exactly the intersection of \( I(Z_1) \)
		      and \( I(Z_2) \).
	\end{enumerate}
\end{remark}

\begin{proposition}
	\label{prop:closure-of-Z-is-V(I(Z))}
	For every \( Z \subseteq \bA^n \), we have:
	\[
		V(I(Z)) = \overline{Z}
	\]
	where the closure is w.r.t. Zariski topology.
\end{proposition}

\begin{proof}
	Recall that by definition of closure, have:
	\[
		\overline{Z} = \bigcap_{Z \subseteq V(J)} V(J)
	\]
	Then:
	\begin{itemize}
		\item \( Z \subseteq V(I(Z)) \implies \overline{Z} \subseteq V(I(Z))\) as \( V(I(Z)) \) is
		      closed set and \( \overline{Z} \) is the smallest closed set containing \( Z \).
		\item To show that \( V(I(Z)) \subseteq \overline{Z} \), need to show that if \( Z\subseteq
		      V(J) \), then \( V(I(Z)) \subseteq V(J) \). See that \( Z \subseteq V(J) \implies J
		      \subseteq I(Z) \implies J \subseteq I(Z) \implies V(J) \supseteq V(I(Z) \).
	\end{itemize}
\end{proof}

\begin{theorem}[\textbf{Hilbert's Nullstellensatz}]
	\label{thm:hilberts-nullstellensatz}
	If \( J \subseteq R = k[x_1,\ldots, x_n] \implies I(V(J)) = \rad(J) \).
\end{theorem}

\begin{remark}
	\leavevmode
	\begin{enumerate}
		\item \( J \subseteq \underbrace{I(V(J))}_{\text{radical ideal}} \) is clear and thus
		      \( \rad(J) \subseteq I(V(J)) \), one direction is quickly yielded, the interesting statement
		      is the converse.
		\item \textbf{Proposition} \ref{prop:closure-of-Z-is-V(I(Z))} +
		      \textbf{Hilbert's Nullstellansatz} \ref{thm:hilberts-nullstellensatz} reveals the following \textbf{mutual order reversing
			      biijections}, which bridge the geometry and algebra and being the most openning and
		      important theorem in \textcolor{blue}{Algebraic Geometry}.
		      \[
			      \begin{tikzcd}[column sep=5em]
				      \begin{matrix}
					      \text{\textcolor{blue}{Geometry}} \\[1ex]
					      \left\{ \begin{matrix} \text{Alg subsets} \\ \text{of } \mathbb{A}^n \end{matrix} \right\}
				      \end{matrix}
				      \arrow[r, shift left=0.6ex, yshift=-1.7ex, "I(-)"] &
				      \begin{matrix}
					      \text{\textcolor{blue}{Algebra}} \\[1ex]
					      \left\{ \begin{matrix} \text{Radical ideals} \\ \text{in } R \end{matrix} \right\}
				      \end{matrix}
				      \arrow[l, shift left=0.6ex, yshift=-1.7ex, "V(-)"]
			      \end{tikzcd}
		      \]
	\end{enumerate}
\end{remark}

\begin{theorem}[\textbf{Weak Nullstellensatz}]
	\label{thm:weak-nullstellensatz}
	Every maximal ideal \( M \) in \( R \) is of the form \( (x_1-a_1, \ldots, x_n -a_n) \) for some
	\( (a_1, \ldots, a_n) \in k^n \).
\end{theorem}

\begin{proof}
	We'll only work with the proof in special case, namely when \( k \) is \textbf{uncountable}, which
	is the case for \( \CC \).

	\textcolor{red}{We will show}: Given \( M \) being maximal ideal of \( R \), there exists \( a_1,\ldots, a_n \), s.t. \( x_i -
	a_i \in M \; \forall \; i \implies \underbrace{(x_1-a_1, \ldots, x_n-a_n)}_{\text{maximal ideal}}
	\subseteq M \implies M = (x_1 - a_1, \ldots, x_n - a_n)\). First notice we have the following
	morphism:
	\[
		k \hookrightarrow R \longrightarrow \qo{R}{M} =: L \text{ being a field}
	\]
	See that \( L \) is a vector space over \( k \): \( L \) is generated by
	\[
		\{\overline{x_1^{a_1}\cdots x_n^{a_n}} \; | \; a_1, \ldots, a_n \in \ZZ_{\geq 0}\}
	\]
	the generation is quite clear if one revisit how \( R \) and thus \( \qo{R}{M} \) are constructed.
	Basically being the linear combination of the monomials in the form of \( x_1^{a_1}\cdots x_n^{a_n}
	\). This implies that \( \dim_{k} L \) is at most countable.

	Now consider the \( k \)-algebra homomorphism:
	\[
		\begin{aligned}
			k[y] & \xrightarrow{\vp} L \\
			y    & \to \overline{x_i}
		\end{aligned}
	\]
	where \( k[y] \) is a domain (not a field!). Now consider the situation of the kernel of \( \vp \):
	\begin{itemize}
		\item If \( \ker(\vp) = 0 \implies \) thus have the injection  \(\text{Frac(k[y])} \hookrightarrow L \). (It being a field,
		      containing a domain, and thus must contain its fraction field).
		      \begin{note}
			      \( \{\frac{1}{y - \lambda} \; | \; \lambda \in k\} \) being a independent set over \( k \).
			      Thus it is uncountable as we assume \( k = \CC \), which leads to contradiction with
			      previous reasoning that \( \dim_k L \) at most countable \( \lightning \).
		      \end{note}
		      \begin{proof}
			      Why is it independent set? Consider
			      \[
				      \begin{aligned}
					      \sum_{i=1}^r c_i \frac{1}{y-\lambda_i}                    & = 0 \\
					      \implies \sum_{i=1}^r c_i \prod_{j \ne i} (y - \lambda_j) & = 0
				      \end{aligned}
			      \]
			      If we make \( y = \lambda_k \), then
			      \[
				      \implies c_k \underbrace{\prod_{j \ne k} (\lambda_k - \lambda_j)}_{\ne 0} = 0
				      \underbrace{\implies}_{\text{by domain}}
				      c_k = 0
			      \]
		      \end{proof}
		      \begin{note}
			      The injection here is important as it basically preserve the relation of dimensionality,
			      which is the key to yield the contradiction. Besides that, the proof is quite like
			      Lagrange interpolations, which is quite intuitive and straightforward.
		      \end{note}
		      \textcolor{red}{Thus there will be no case that \( \ker(\vp) = \{0\} \)}.
		\item If \( \ker(\vp) \ne \{0\}\implies \exists \; f \in k[y] \), s.t. \( f(\overline{x_i})
		      = 0 \) where the equality holds in target field \( L \). Since \( k \) is algebraically closed, can factor \( f \) as \(
		      c(y-\lambda_1)\cdots (y-\lambda_m) \) where \( c, \lambda_1,\ldots, \lambda_m \in k \).
		      Thus
		      \[
			      c(\overline{x_i} - \lambda_1) \cdots (\overline{x_i} - \lambda_m) = 0 \implies
			      \overline{x_i} = \lambda_j \in k \text{ for some } j \text{ as }L \text{ is a field}
		      \]
		      thus \( \overline{x_i} \in k \implies \exists \; a_i \in k\), s.t. \( \overline{x_i} =
		      \overline{a_i} \implies x_i - a_i \in M \). Doing mapping repeatedly for all \( x_i \)
		      will yield the desired result as we stated at the beginning of the proof.
	\end{itemize}
\end{proof}

\begin{proof}[\textbf{Proof of Hilbert's Nullstellensatz
			\ref{thm:hilberts-nullstellensatz}} (via
		Rabinowitsch's Trick)]
	Suppose \( f \in I(V(J)) \), and we want to see that \( \exists \; d > 0 \), s.t. \( f^d \in J \).
	Let's take \( S = R[y] \supseteq J' \) being the ideal generated by \( J \) and \( g \) where
	\[
		g = 1 - fy
	\]
	\begin{claim}
		\( V(J') = \emptyset \).
	\end{claim}
	If \( a' = (a,\lambda) \in V(J') \) where \( a \in k^n \) and \( \lambda\in k \implies a \in V(J)
	\implies f(a) = 0\). Thus
	\[
		g(a') = 1 - f(a) \lambda = 1 \ne 0
	\]
	leading to contradiction \( \lightning \) as \( a' \) is taken as the Nullstellen of \( g \) and thus \( V(J') =
	\emptyset \).

	If \( J' \ne S \), then \( J' \subseteq M \) where \( M \) be some maximal ideal. Then by
	\textbf{Weak Nullstellensatz}
	\ref{thm:weak-nullstellensatz}, have \( V(M) \ne
	\emptyset \implies\) \( V(J')\ne \emptyset \) \( \lightning \). Hence \( J' = S \implies \)
	\[
		\begin{aligned}
			\exists \;     & f_1,\ldots, f_m \in J               \\
			\exists \;     & g_1,\ldots, g_m; h \in S            \\
			\text{s.t. } 1 & = f_1g_1 + \cdots f_m g_m + h(1-fy)
		\end{aligned}
	\]
	Let \( T = \{f^p \; | \; p \geq 0\} \) and take advantadge of the following ring homomorphism:
	\[
		\begin{aligned}
			R[y] & \to R_f = T^{-1} R  \\
			y    & \mapsto \frac{1}{f}
		\end{aligned}
	\]
	In \( R_f \), have:
	\[
		1 = \sum_{i=1}^m f_i g_i (x_1,\ldots, x_n, \frac{1}{f}) \quad \text{$y$ here was localized into
			$\frac{1}{f}$}
	\]
	If \( d \gg 0 \), then \( f^d = f^d \cdot \) RHS \( \in J \). Note that \( f^d \cdot \) RHS is to
	erase all the \( f \) who is on the denominator, and thus \( f^d \cdot \) RHS is a polynomial in
	\( R \) and it is in \( J \) by the ideal property.
\end{proof}
\textbf{Hilbert's Nullstellensatz} is the beginning of Alegbraic Geometry, which allow us to use
algerba to explain the property of geometry and vice versa.

