\section{Quotient Rings}

In this section we construct quotient rings. Main heuristics is to follow the construction of quotient groups while maintainning the compatibility with multiplication, in particular, with \textbf{ring homomorphism}.

Let \( (R, +, \cdot) \) be a ring, if \( I \subseteq R \) be subgroup, then \( I \) is automatically normal since \( (R, +) \) is abelian. Thus we can construct \( \qo{R}{I} \) as a group:
\[
	\qo{R}{I} \coloneqq \qo{R}{\equiv \bmod I} \quad a \equiv b \bmod I \text{ if } a - b \in I
\]

Write \( a+ I \) or simply \( \overline{a} \) or \( [a] \) for the image of \( a\in R \) in \( \qo{R}{I} \). The group structure is \textbf{defined} s.t.:
\[
	\begin{aligned}
		\pi : R & \to \qo{R}{I} \\
		a       & \mapsto a+I
	\end{aligned}
\]

is \textbf{group homomorphism}. which is:
\[
	\overline{a} + \overline{b} =\overline{a+b}
\]

We then want to see that \( \qo{R}{I} \) to be not just a group, but make it a \textbf{ring}, which is: \( \pi \) to be a \textbf{ring homomorphism}.

Since \( \ker(\pi) = I \), for the above to work, we need \( I\subseteq R \) is a \textbf{2-sided ideal}. So let's just assume \( I \) is a 2-sided ideal.

Since we want \( \pi \) to be a ring homomorphism, we have to define multiplication on \( \qo{R}{I} \), which is by the most obvious way:
\[
	\overline{a} \cdot \overline{b} = \overline{ab}
\]

The \textbf{key point} here is then to show that it is \textbf{well-defined}. And we need: if \( \overline{a} = \overline{a'}, \overline{b} = \overline{b'}  \implies \overline{ab} = \overline{a'b'}\).

We know that \( a - a' \in I, b - b' \in I \) and we want \( ab - a'b' \in I \), which is:
\[
	\begin{aligned}
		ab - a'b' & = (ab-ab') + (ab' - a'b;)                                                                                                                                         \\
		          & = \underbrace{a\underbrace{(b-b')}_{\in I}}_{\in I \text{ since left ideal}} + \underbrace{\underbrace{(a-a')}_{\in I}b'}_{\in I \text{ since right ideal}} \in I
	\end{aligned}
\]

Once we know that multiplication is well-defined, need the following:
\begin{itemize}
	\item multiplication is associative.
	      \[
		      \underbrace{\underbrace{( \overline{a} \overline{b} ) \overline{c}}_{=\overline{ab} \overline{c} = \overline{(ab)c}} = \underbrace{\overline{a} ( \overline{b} \overline{c} )}_{=\overline{a}\overline{bc} = \overline{a(bc)}}}_{\text{by associativity in } R} \quad \forall \; \overline{a}, \overline{b}, \overline{c} \in \qo{R}{I}
	      \]
	\item distributivity holds by similar argument as above.
	\item identity element for multiplication.
	      \[
		      \begin{aligned}
			      \overline{1} \overline{a} & = \overline{1a} = \overline{a} \\
			      \overline{a} \overline{1} & = \overline{a1} = \overline{a}
		      \end{aligned}
	      \]
	\item if \( R \) is commutative, then \textbf{so is} \( \qo{R}{I} \).
\end{itemize}

The \textbf{upshot} is: \( \qo{R}{I} \) is a ring and \( \pi: R \to \qo{R}{I} \) is a ring homomorphism, note that \( \pi(1_R) = 1_{\qo{R}{I}} \).

\begin{proposition}[\textbf{Universal Property of Quotient Rings}]
	Suppose \( R, I \) are as before, let \( f: R \to S \) be a ring homomoprhism, s.t. \( I\subseteq \ker(f) \). There is a \textbf{unique} ring homomophrism \( \overline{f}: \qo{R}{I} \to S \), s.t. the following diagram is \textbf{commutative}:
	\[
		\begin{tikzcd}
			R \arrow[r, "\pi"] \arrow[d, "f"'] & \qo{R}{I} \arrow[ld, "\bar{f}", dashed] \\
			S &
		\end{tikzcd}
	\]

	which is \( f = \overline{f}\circ \pi \).
\end{proposition}

The main idea of the proof is to inherit from our idea for universal property of quotient groups and see that is compatible with ring multiplication.
\begin{proof}
	The condition \( f = \overline{f} \circ \pi \iff \overline{f}( \overline{a} ) = f(a) \; \forall \; a \in R \), this implies uniqueness, since \( \pi \) is surjective, as it is explicitely defined for the whole domain \( \qo{R}{I} \).

	By the corresponding results for groups, there exists \( \overline{f}: \qo{R}{I} \to S \) to be group homomorphism, s.t. \( f = \overline{f}\circ \pi \). Hence, it is enough to show:
	\begin{itemize}
		\item \( \overline{f}(u\cdot v)=\overline{f}(u)\overline{f}(v) \; \forall \; u,v\in \qo{R}{I} \).
		\item \( \overline{f}(1_{\qo{R}{I}}) = 1_S \).
	\end{itemize}

	the second assertion follows directly:
	\[
		\overline{f}(1_{\qo{R}{I}}) = \overline{f}(\pi(1_R)) = f(1_R) = 1_S
	\]

	for the first assertion, write \( u = \overline{a}, v=\overline{b} \) for some \( a,b \in R \), then:
	\[
		\overline{f}(uv) = \overline{f} (\overline{ab}) = f(ab) = f(a) \cdot f(b) = \overline{f}(\overline{a}) \overline{f} (\overline{b}) = \overline{f}(u) \overline{f}(v)
	\]
\end{proof}

\section{Isomorphism Theorem}

Follow similarly with group, there is also corresponding isomorphism for rings. One should notice that the quotient ring is quite restricted since it requires \( I \) to be a \textbf{two-sided ideals}, not either left or right ideal.

\begin{theorem}[\textbf{Fundemental Isomorphism Theorem}]
	\label{thm:fundemental-isomorphism}
	If \( f: R \to S \) is a \textbf{surjective} ring homomorphism, and \( I = \ker(f) \implies S \cong \; \qo{R}{I} \).

	\begin{note}
		Note that the theorem implies that \( \im(f) \cong \; \qo{R}{I} \).
	\end{note}
\end{theorem}

\begin{remark}
	If \( f \) be arbitrary ring homomorphism, then \( \im(f) \subseteq S \) is a subring.
\end{remark}

\begin{proof}
	Since \( I = \ker(f) \implies I\) is a two-sided ideal. Apply the universal property of \( \qo{R}{I} \), the following diagram is commutative:
	\[
		\begin{tikzcd}
			R \arrow[r, "\pi"] \arrow[d, "f"'] & \qo{R}{I} \arrow[ld, "\bar{f}", dashed] \\
			S &
		\end{tikzcd}
	\]

	there exists a unique ring homomorphism \( \overline{f}: \qo{R}{I} \to S \), s.t. \( \overline{f}\circ \pi =  f \). In the context of group, we've shown that \( \overline{f} \) is a group isomorphism, so \( \overline{f} \) is bijective, thus it is a ring isomorphism.
\end{proof}

We then consider the analog of the third isomorphism for groups. We want to describe the left/right/two-sided ideals of \( \qo{R}{I} \) in terms of the ones for \( R \), and in fact we have the following proposition.
\begin{proposition}
	We have an order preserving bijection:
	\[
		\begin{tikzcd}[column sep=huge, row sep=large]
			\left\{ \begin{tabular}{c} left/right/2-sided \\ ideals in $\qo{R}{I}$ \end{tabular} \right\}
			\arrow[r, shift left=1.5ex, "J \longrightarrow \pi^{-1}(J)"]
			&
			\left\{ \begin{tabular}{c} left/right/2-sided \\ ideals of $R$ containing $I$ \end{tabular} \right\}
			\arrow[l, shift left=1.5ex, "\pi(I') \longleftarrow I' \supseteq I"]
		\end{tikzcd}
	\]

	where
	\[
		\pi: R \to \qo{R}{I}
	\]
\end{proposition}

\begin{proof}
	We have already seen these two maps given \textbf{mutual inverses} for corresponding in groups, to conlcude, we only need to show:
	\begin{itemize}
		\item \( J \subseteq \qo{R}{I} \) is a left/right/two-sided ideal, then so is \( \pi^{-1}(J) \subseteq R \).
		\item \( I'\subseteq R \) is a left/right/two-sided ideal, then so is \( \pi(I') \).
	\end{itemize}

	It will be proved in homework.
\end{proof}

\begin{notation}
	if \( I' \subseteq I \) be ideal, we denote \( \pi(I') \) by \( \qo{I'}{I} \).
\end{notation}

\begin{theorem}[\textbf{Third Isomoprhism Theorem}]
	If \( R \) is a ring and \( I\subseteq I' \) are two-sided ideals, then:
	\[
		\qo{\qo{R}{I}}{\qo{I'}{I}} \cong \; \qo{R}{I'}
	\]

	Note that quotient rings doesn't make sense when \( I' \) is left ideal or right ideal.
\end{theorem}

\begin{proof}
	By the universal property of \( \qo{R}{I} \) for \( R \xrightarrow{p} \qo{R}{I'} \), there exists a unique \( \overline{p}: \qo{R}{I} \to \qo{R}{I'} \), s.t. \( p(a+I) = a+I' \; \forall \; a\in R \).

	Easy to see that \( \overline{p} \) is surjective and \( \ker(\overline{p}) = \qo{I'}{I}\) as we've concluded in context of groups, then by the fundemental isomorphism theorem, yields the result.
\end{proof}

\begin{eg}
	Let \( n \in \ZZ_{>0} \), in \( \ZZ \), we have ideal:
	\[
		(n)\coloneqq \{nk \; | \; k \in \ZZ\}
	\]

	then \( \qo{\ZZ}{(n)} \) is exactly \( \qo{\ZZ}{n\ZZ} \). \( d\in \ZZ_{>0}, \; (d) \supseteq (n) \iff d \big| n \). We have an ideal:
	\[
		(\overline{d}) \coloneqq  \{\overline{d} a \; | \; a\in \qo{\ZZ}{n\ZZ}\} = \qo{(d)}{(n)}
	\]

	and the theorem implies:
	\[
		\qo{\bace{\qo{\ZZ}{n\ZZ}}}{(\overline{d})} \cong \; \qo{\ZZ}{d\ZZ}
	\]
\end{eg}
