\section{Noetherian Rings}

In this section we focus on the context of \( R \) being any ring instead of being commutative ring.

\begin{definition}[\textbf{Noetherian Rings and Artirian Rings}]
	Let \( R \) be any ring. Then \( R \) is \underline{left/right}  \underline{Noetherian ring} if there exists \textbf{no} strictly increasing sequence:
	\[
		I_1 \subsetneq I_2 \subsetneq \cdots \text{ of left/right ideals. }
	\]
	\( R \) is \underline{left/right Artirian ring} if there exists \textbf{no} strictly decreasing sequence
	\[
		I_1 \supsetneq I_2 \supsetneq \cdots \text{ of left/right ideals.}
	\]
\end{definition}

\begin{note}
	Aritirian ring is a stronger property, most of the rings doesn't satisfy the property of artirian ring.
\end{note}

\begin{eg}
	\leavevmode
	\begin{enumerate}
		\item A field is both noetherian and artirian.
		\item Every finite ring is both noetherian and artirian.
		\item \( \ZZ \) is not a artirian ring, because there exists the following strictly decreasing ideal sequence:
		      \[
			      2\ZZ \supsetneq 2^2 \ZZ \supsetneq 2^3 \ZZ \supsetneq \cdots
		      \]
	\end{enumerate}
\end{eg}

\begin{proposition}
	\label{prop:noetherian-equiv}
	For a ring \( R \), \textbf{TFAE}:
	\begin{enumerate}
		\item \( R \) is left/right Noetherian ring.
		\item Every non-empty family \( \cI \) of left/right ideals of \( R \) contains a maximal element. The maximal element namely is:
		      \[
			      \exists \; I \in \cI \; \text{s.t. if } I \subseteq J, \; J \in \cI \implies I=J
		      \]
		\item Every left/right ideal of \( R \) is finitely generated. Namely:
		      \[
			      \exists \; a_1, \ldots, a_n \in R, \text{ s.t. } I = \sum_{i=1}^n Ra_i \quad \bace{I = \sum_{i=1}^n a_i R}
		      \]
	\end{enumerate}
\end{proposition}

\begin{proof}
	\leavevmode
	\begin{itemize}
		\item \( (2)\implies (1) \): This is clear as one cannot find family of ideals that form a strictly increasing sequence family given the condition of existence of maximal element.
		\item \( (1)\implies (2) \): One can reason by contraposition, non-existing maximal element means we can always find ideal that is strictly bigger than the current one, and thus construct a infinite increasing sequence, fail the property of Noetherian ring.
		\item \( (1)\implies (3) \): Given a left ideal \( I \) that is not finitely generated. We can construct inductively \( a_1, a_2, \ldots \in i \), s.t.:
		      \[
			      a_{n+1} \not\in \sum_{i=1}^n Ra_i \; \forall \; n
		      \]
		      This implies that the sequence of ideals \( \bace{\sum_{i=1}^n Ra_i} \) is strictly increasing and infinite, thus fail the property of Noetherian ring.
		\item \( (3)\implies (1) \): Suppose we have \( I_1\subseteq I_2\subseteq I_3 \subseteq \cdots \) is a sequence of left ideals. Let:
		      \[
			      I \coloneqq  \bigcup_{n\geq 1} I_n
		      \]
		      we've seen before in the proof of \textbf{Theorem} \ref{thm:maximal}, \( I \) is also a left ideal given that \( \{I_\alpha\} \) is a total ordered set. Now \textbf{by (3)}, \( I \) is finitely generated, so we can write:
		      \[
			      I = \sum_{i=1}^n Ra_i \text{ for some } a_1,\ldots, a_n \in R
		      \]
		      Suppose that \( a_i \in I_{r_i} \) for \( 1\leq i \leq n \), let \( r = \max\{r_i\} \), this implies that:
		      \[
			      \sum_{i=1}^n Ra_i \subseteq I_r \subseteq I = \sum_{i=1}^n Ra_i \quad \forall \; s \geq r
		      \]
		      thus \( I_s = I_r \; \forall \; s\geq r \). So we see that the sequence \textbf{stablize to some ideal}, leading to the property of Noetherian ring.
	\end{itemize}
\end{proof}

The following theorem gives us a basic idea that in fact a lot of rings a actually Noetherian, because the property can inherit from some other ring. So we can build up Noetherian ring from the existing one.

\begin{theorem}[\textbf{Hilbert's Basis Theorem}]
	\label{thm:hilbert-basis}
	If \( R \) is a \textbf{Noetherian commutative ring}, then \( R[X] \) is also a Noetherian ring.
\end{theorem}
In fact this proof is quite influentiall and this is actually the first result in Commutative Algebra. Commutative property here basically only help us construct the polynomial ring.

\begin{proof}
	We will show basically every ideal \( I \subseteq R[x] \) is finitely generated, and yield that \( R[x] \) is Noetherian by \textbf{Proposition} \ref{prop:noetherian-equiv}.

	Suppose \( I \) is not finitely generated. Then we can inductively construct a sequence of element as follow, each time of iteration, the condition on the degree of such element is a little bit stronger than before.
	\[
		\begin{aligned}
			I & \ne \{0\} \implies \text{chose } f_1 \in I \backslash \{0\} \text{ of minimal degree } d_{1} \\
			I & \ne (f_1) \implies \text{chose } f_2 \in I \backslash (f_1) \text{ of minimal degree } d_{2} \\
		\end{aligned}
	\]
	We repeat this to construct \( f_1,f_2,\ldots \), s.t.:
	\[
		\forall n \geq 0, \; f_{n+1} \in I \backslash (f_1,f_2,\ldots, f_n)
	\]
	and \(\deg(f_{n+1} = d_{n+1}) \) is minimal among such \( f_{n+1} \). The reason why we can construct \( f_i \) in this way is that \( I \) is not finitely generated, so there is always some element between \( (f_1,\ldots,f_n) \subset I \).
	\begin{note}
		By constuction and \textbf{minimality} of \( d_n \), we have:
		\[
			d_1\leq d_2\leq \cdots
		\]
	\end{note}
	We then want to make contradiction to this minimality. For every \( n\geq 1 \), let's write:
	\[
		f_n = a_n x^{d_n} + \text{ lower degree monomials}
	\]
	Let \( J \subseteq R \) be the ideal generated by \( \{a_n \; | \; n\geq 1\} \). Since \( R \) is a Noetherian ring, \( J \) is then \textbf{finitely generated}. So we can write:
	\[
		J = (b_1, \ldots, b_k)
	\]
	We can write each \( b_i \) as an linear combination of \( a_j \)'s, by the fact that \( J \) be the ideal generated by the whole sequence \( \{a_n \; | \; n\geq 1\} \). Namely:
	\[
		b_i = \sum_{j=1}^{m_i} \lambda_{ij}a_j \quad \lambda_{ij} \in R
	\]
	Now if we let \( m = \max\{m_i\} \), then \( (a_1,\ldots, a_m) \supseteq (b_1,\ldots, b_k) = (a_1,a_2,\ldots) \), and it is clear that \( (a_1,\ldots, a_m) \subseteq (b_1,\ldots, b_k) \). Thus we conclude.
	\begin{conclusion}
		\( J = (a_1,\ldots, a_m) \).
	\end{conclusion}

	Thus see that \( a_{m+1}\in (a_1,\ldots, a_m) \implies \) we can write \( a_{m+1} = \lambda_1a_1 + \cdots + \lambda_m a_m \). One can then define:
	\[
		\begin{aligned}
			g & = \underbrace{f_{m+1}}_{=a_{m+1}x^{d_{m+1}} + \text{l.d.m.}} - \sum_{i=1}^m \underbrace{\underbrace{\lambda_i x^{\overbrace{d_{m+1}-d_i}}}^{\text{\textbf{by non-dec.}}}_{\in \mathbf{R[x]} }f_i}_{\lambda_i a_i x^{d_{m+1}} + \text{ l.d.m. \todo{l.d.m. stands for lower degree monomials.}}} \\
			g & = f_{m+1} - \sum_{i=1}^m \lambda_i x^{d_{m+1}-d_i} f_i
		\end{aligned}
	\]
	\begin{note}
		\leavevmode
		\begin{enumerate}
			\item \( \deg(g) < d_{m+1} \) \hypertarget{contra}{$(\star)$}.
			\item \( g\in I \) and \( g\not\in (f_1,\ldots, f_n) \), otherwise, see that:
			      \[
				      f_{m+1} = g + \sum_{i=1}^m \lambda_i x^{d_{m+1}-d_i} f_i \in (f_1, \ldots, f_m)
			      \]
			      which is \textbf{not} ok.
		\end{enumerate}
	\end{note}

	Thus see that \hyperlink{contra}{\( (\star) \)} implies a contradiction with minimality of \( d_{m+1} \).
\end{proof}

\begin{corollary}
	If \( \KK \) is a field, then \( \KK[x_1,\ldots, x_n] \) is Noetherian ring for all \( n\geq 1\).
\end{corollary}

\begin{proof}
	Basically use \textbf{Theorem} \ref{thm:hilbert-basis} and induction on \( n \) and the fact that \( \KK \) is a Noetherian ring.
\end{proof}

\section{PIDs and Euclidean Domains}

\begin{definition}[\textbf{PID}]
	Let \( R \) be a domain, then \( R \) is a \underline{Principal ideal domain (PID)} if every ideal in \( R \) is \underline{principal}: namely it is of the form \( (a) \) for some \( a\in R \).
\end{definition}

\begin{eg}[\textbf{Non-PID Example}]
	\leavevmode
	\begin{enumerate}
		\item \( (2,x) \subseteq \ZZ[x] \) is not a principal ideal, and thus \( \ZZ[x] \ne \) PID.
		      \begin{proof}
			      Suppose that \( (2,x) = (f) \) for some \( f\in \ZZ[x] \), then \( 2\in (f) \implies 2 = fg\) for some \( g\in \ZZ[x] \implies \deg(f) = 0 \). Thus see that \( f = n \in \ZZ\backslash \{0\} \), and see that \( x = ng \implies n = \pm 1 \) by considering the fact that the coefficient of \( x \) is \( 1 \), and in the domain \( \ZZ \), the only way to get \( 1 \) is by multiplying \( \pm1 \).

			      Hence \( (2,x) = \ZZ[x] \) by the fact that \( \pm1 \) is in the ideal. Then it means that there exists \( P, Q \in \ZZ[x] \), s.t. \( 1 = 2 P + x Q \).
			      \[
				      \begin{aligned}
					      P & = a_0 + a_1 x + \cdot \\
					      Q & = b_0 + b_1x + \cdots
				      \end{aligned}
			      \]
			      \( \implies 1 = 2a_0 \), but it cannot happen since \( a_0 \in \ZZ \).
		      \end{proof}
		\item If \( \KK \) is a field, then \( (x,y) \subseteq \KK[x,y] \) is not finitely generated. \todo{exer!}
	\end{enumerate}
\end{eg}

\begin{definition}[\textbf{Euclidean Domain}]
	A domain \( R \) is an \underline{Euclidean Domain} if there eixsts \( N: R\backslash\{0\} \to \ZZ_{\geq 0} \) being \textbf{arbitrary} functions, such that, \( \forall \; a,b \in R \) with \( b\ne 0 \), there exists \( q,r\in R \) (\textbf{not necessarily unique}), such that:
	\begin{itemize}
		\item \( a = bq + r \).
		\item Either \( r= 0 \) or (\( r\ne 0 \) and \( N(r)<N(b) \)).
	\end{itemize}
\end{definition}

\begin{proposition}
	Every Euclidean Domain \( R \) is a PID.
\end{proposition}

\begin{proof}
	Let \( I \subseteq R \) be an ideal, If \( I = \{0\} \), clearly that \( I \) is principl. If \( I \ne \{0\} \) and \( N \) is as in the definition of Euclidean domain. Let \( b \in I \backslash\{0\} \) be s.t. \( N(b) \) is minimal.
	\begin{claim}
		We claim that \( I = (b) \).
	\end{claim}
	``\( \supseteq \)'' part is clear as \( b\in I \). For ``\( \subseteq \)'' part, Suppose that \( a\in I \), with \( b\ne 0 \), then \( \exists \; q,r \in R \), s.t. \( a = bq + r \) and:
	\begin{itemize}
		\item \( r = 0\implies a\in (b) \) which is ok.
		\item \( r\ne 0 \) and \( N(r) < N(b) \), thus \( r = a-bq \in I \) by the ideal property, which contradict to the minimality assumption of \( N(b) \) $\lightning$. So this case cannot happen.
	\end{itemize}
\end{proof}

\begin{eg}
	\leavevmode
	\begin{enumerate}
		\item \( \ZZ \) is an Euclidean domain, with:
		      \[
			      \begin{aligned}
				      N: \ZZ\backslash \{0\} & \to \ZZ_{\geq 0} \\
				      N(a)                   & = \abs{a}
			      \end{aligned}
		      \]
		      \begin{proof}
			      Suppose \( b \in \ZZ\backslash\{0\} \), then there are two cases:
			      \begin{itemize}
				      \item \( b>0 \): we know \( a = qb+r \), \( 0\leq r < b = N(b) \) for some \( q,r \in \ZZ \) with the usual division algorithm.
				      \item \( b<0 \): do the same for \( -a, -b \). Have \( -a = (-b)q+r \iff a = bq - r \) for \( 0\leq r < -b \), with \(N(r) \abs{-r} = r < -b = \abs{b} = N(b) \).
			      \end{itemize}
		      \end{proof}
		\item \( \ZZ[i] \) is a Euclidean domain, with:
		      \[
			      \begin{aligned}
				      N: \ZZ[i] & \to \ZZ_{\geq 0} \\
				      N(a+bi)   & = a^2 + b^2
			      \end{aligned}
		      \]
		      \begin{proof}
			      Let \( a+bi, c+di \ne 0 \in \ZZ[i] \), in \( \QQ[i] \), have:
			      \[
				      \begin{aligned}
					      \frac{a+bi}{c+di} & = p+qi \quad p,q \in \QQ                                                                                         \\
					                        & = (\alpha + \beta i) + (\gamma + \delta i)                                                                       \\
					                        & \text{where } \alpha, \beta \in \ZZ, \; \gamma, \delta \in \QQ, \; \abs{\gamma}, \abs{\delta } \leq \frac{1}{2}.
				      \end{aligned}
			      \]
			      This implies that:
			      \[
				      \begin{aligned}
					      \underbrace{a+bi}_{\in \ZZ[i]} & = \underbrace{(c+di)}_{\in \ZZ[i]} \cdot \underbrace{(\alpha + \beta i)}_{\in \ZZ[i]} + r \\
					                                     & \text{where } r = (c+di)(\gamma + \delta i)                                               \\
					      \implies r                     & \in \ZZ[i]
				      \end{aligned}
			      \]
			      thus:
			      \[
				      N(r) = N(c+di) \cdot \underbrace{N(\gamma + \delta_i)}_{=\gamma^2 + \delta^2 \leq \frac{1}{4} + \frac{1}{4} = \frac{1}{2}} \quad \text{by \hyperlink{eq:hom}{\( (\star \star) \)}}
			      \]
			      thus either \( r = 0 \) or \( N(r) < N(c+di) \) since \( N(c+di)\ne 0 \).
		      \end{proof}
		\item If \( \KK \) is a field, then \( \KK[x] \) is an Euclidean domain, with:
		      \[
			      \begin{aligned}
				      N: \KK[x] \backslash \{0\} & \to \ZZ_{\geq 0} \\
				      N(f)                       & = \deg(f)
			      \end{aligned}
		      \]
		      We verify the condition by showing the following proposition.
		      \begin{proposition}
			      Let \( R \) be any commutative ring, \( f,g \in R[x] \) and \( g \ne 0 \). Let
			      \[
				      g = a_n x^n + \cdots + a_1 x + a_0
			      \]
			      s.t. \( a_n \) is \textbf{invertible}, then there exists \textbf{unique} \( q,r \in R[x] \), s.t. \( f = gq + r \) and either \( r=0 \) or \( r\ne 0 \) and \( \deg(r) < n = \deg(g) \).
		      \end{proposition}
		      \begin{proof}[\textbf{Proof of Existence is Enough}]
			      Proceed the proof by contradiction. If for given \( g \), there are \( f \)'s that don't satisfy this condition. Let's choose such \( f \) of minimal degree, clearly we have \( m := \deg(f) \geq n \implies f = b_m x^m + \) lower order term. Consider
			      \[
				      f' = f - b_m a_n^{-1} x^{m-n}g
			      \]
			      see that \( \deg(f') < \deg(f)\implies \exists \; q',r' \), s.t. \( f' = q'g + r' \implies f = \bace{b_m a^{-1}_n x^{m-n}+q'}g + r' \), which leads to contradiction $\lightning$, as we see that since \( f \) is minimal case that don't satisfy the condition, meaning we have either \( r'=0 \) or \( r'\ne 0 \) and \( \deg(r) < n = \deg(g) \).
		      \end{proof}
		      \begin{note}
			      To show the \underline{uniqueness}, show that \( \deg(gh) = \deg(g) + \deg(h) \; \forall \; h\). We've proceed this for the case of domain, and it is crucial that \( a_n \) is invertible here.
		      \end{note}
	\end{enumerate}
\end{eg}

\begin{remark}
	It is very hard to say a domain is not an euclidean domain, because it's very hard to tell that \( N \) doesn't exists for sure.
\end{remark}
