\begin{remark}
	If \( R \) ring, \( M,N \) are (left) \( R \)-modules, then
	\begin{itemize}
		\item \( \Hom_R (M,N) \) is an abelian group with operation defined:
		      \[
			      (f+g)(m) := f(m) + g(m), \quad \forall m \in M
		      \]
		      one can check that \( f + g \in \Hom_R (M,N) \) and this operation makes \( \Hom_R (M,N) \)
		      an abelian group. The \( 0 \) element is given by the constant \( 0 \) map.
		\item We have
		      \[
			      \begin{aligned}
				      \Hom_R (M,N) \times \Hom_R(N,P) & \to \Hom_R (M,P) \\
				      (f,g)                           & \mapsto g\circ f
			      \end{aligned}
		      \]
		      is additive in each factor. We want to make \( \Hom_R (M,N) \) an \( R \)-module, we
		      define:
		      \[
			      (af)(u) \coloneqq  af(u) = f(au) \quad \forall \; a\in R, u\in M, f \in \Hom_R (M,N)
		      \]
		      It is easy to see that \( af \) is additive, and need to see it is compatible with scalar
		      multiplication:
		      \[
			      (af)(bu) \xlongequal{?} b(af)(u) \quad \forall \; b \in R
		      \]
		      see that
		      \[
			      \begin{aligned}
				      \text{LHS} & = af(bu) = (ab)f(u) \\
				      \text{RHS} & = (ba) f(u)
			      \end{aligned}
		      \]
		      If \( R \) is commutative, thus making LHS \( = \) RHS, and makes \( \Hom_R (M,N) \) an \(
		      R\)-module.
		\item Let \( R \) being commutative, we have:
		      \[
			      \begin{aligned}
				      \Hom_R (R,M) & \cong \; M   \\
				      f            & \mapsto f(1)
			      \end{aligned}
		      \]
		      which is isomorphism as \( R \)-module. Note that this is because if we choose \( f: R \to
		      M\) being morphism of \( R \)-modules, then \( f(a) = af(1) \; \forall \; a\in R \). And
		      thus the whole \( M \) is generated by \( f(1) \).
	\end{itemize}
\end{remark}

\section{Operations on Submodules}
As submodules can be view as somekind of generalization of ideals, the construction in this section
is pretty much like what we have done for the operation for ideals.

\subsection{Intersection}

\begin{definition}
	If \( (N_i)_{i\in I} \) be family of submodules of \( M \), then
	\[
		\bigcap_{i\in I} N_i \subseteq M
	\]
	is a submodule, which is easily checked.
\end{definition}

\subsection{Submodule Generated by a Subset}

\begin{definition}
	If \( A \subseteq M \) be any subset of the module, define:
	\[
		\langle A \rangle := \bigcap_{\substack{N \subseteq M \\ N \text{ is a submodule} \\ A \subseteq
				N}} N
	\]
	this is the smallest submodule of \( M \) conatining \( A \), and is called the submodule
	generated by \( A \). There is also a more \textcolor{blue}{explicit description} of \( \angl{A} \)
	\[
		\angl{A} = \brac{\sum_{i=1}^r \lambda_i x_i \; | \; r \geq 0, x_i \in A, \lambda_i \in R
			\forall \; i}
	\]
	\begin{proof}
		One can easily check that \( RHS \) is indeed a submodule, and it contains \( A \), so it is
		immediate that \( \angl{A} \subseteq \) RHS. If \( A \subseteq M' \) where where \( M' \) being
		submodule of \( M \implies \) RHS \( \subseteq M' \implies \) RHS \( \subseteq \angl{A} \).
	\end{proof}
\end{definition}

\begin{note}
	\leavevmode
	\begin{itemize}
		\item \( A \subseteq M \) is a \underline{system of generators} of \( M \) if \( \angl{A} = M \).
		\item \( M \) is called \underline{finitely generated} if it has a finite \underline{system of
			      generators}.
	\end{itemize}
\end{note}

\subsection{The Sum of Submodules}

\begin{definition}
	Given a family \( (M_i)_{i\in I} \) of submodules of \( M \), define the sum of those submodules
	\[
		\sum_{i\in I} M_i \coloneqq  \angl{\; \bigcup_{i\in I} M_i \; }
	\]
	one can check that it also has a expclit description:
	\[
		\sum_{i\in I} M_i = \brac{\sum_{i\in I} x_i \; | \; x_i \in M_i \; \forall \; i \in I \text{ and only
				finitely many $x_i$ are non-zero.}}
	\]
\end{definition}

\begin{notation}
	If \( I = \{1,\ldots, n\} \), we write \( M_1 + \cdots + M_n \).
\end{notation}

\begin{theorem}[\textbf{Second Isomorphism Theorem}]
	If \( M_1, M_2 \subseteq M \) are submodules, then:
	\[
		\qo{M_1}{M_1 \cap M_2} \cong \; \qo{M_1 + M_2}{M_2}
	\]
\end{theorem}

\begin{proof}
	Consider the composition defined as follows:
	\[
		f: M_1 \xhookrightarrow{i} M_1 + M_2 \xrightarrow{\pi} \qo{M_1 + M_2}{M_2}
	\]
	\begin{itemize}
		\item This is surjective: given every element in \( \qo{M_1 + M_2}{M_2} \) is of the form \(
		      \pi(u_1 + u_2) \) with \( u_i \in M_i \). But
		      \[
			      \pi(u_1 + u_2) = \pi(i(u_1)) = \pi(u_1) = f(u_1)
		      \]
		\item \( \ker(f) = \{u \in M_1 \; | \; i(u) \in M_2\} = M_1 \cap M_2 \).
	\end{itemize}
	and by first isomorphism theorem, yield:
	\[
		\qo{M_1}{M_1 \cap M_2} = \qo{M_1}{\ker(f)} \cong \; \qo{M_1 + M_2}{M_2}
	\]
\end{proof}

\subsection{Internal Direct Sums}

\begin{definition}
	If \( M_1, M_2 \subseteq M \) being submodules, we have a morphism of \( R \)-modules
	\[
		\begin{aligned}
			f: M_1 \oplus M_2 & \longrightarrow M                        \\
			(u_1, u_2)        & \longmapsto u_1 + u_2 \qquad u_i \in M_i
		\end{aligned}
	\]
	Say \( M \) is the \underline{internal direct sum} of \( M_1, M_2 \) (write \( M = M_1 \oplus M_2 \))
	if \( f \) is an \textbf{isomorphism}. Which is equivalent to say
	\begin{itemize}
		\item \( f \) is surjective, \( \iff \) \( M = M_1 + M_2 \).
		\item \( f \) is injective, \( \iff \)
		      \[
			      \begin{aligned}
				      \ker(f) & = \{0\}                                     \\
				              & = \{(u,v) \; | \; u_i \in M_i, u_2 = -u_1\} \\
				              & = \{(x,-x)\; | \; x\in M_1 \cap M_2\}       \\
				              & \iff M_1 \cap M_2 = \{0\}
			      \end{aligned}
		      \]
	\end{itemize}
	\begin{conclusion}
		\leavevmode
		\[
			M = M_1 \oplus M_2 \iff
			\begin{cases}
				M_1 + M_2 = M \\
				M_1 \cap M_2 = \{0\}
			\end{cases}
		\]
	\end{conclusion}
\end{definition}

\subsection{Functors Between \( R \)-\underline{mod} and \( \qo{R}{I}\)-\underline{mod}}

Suppose \( I \subseteq R \) be a 2-sided ideal, and \( M \) is a left \( R \)-module, s.t. \(
\textcolor{red}{\forall \; a \in I, u\in M}, au = 0 \).

\begin{claim}
	We have a natural structure of \( \qo{R}{I} \)-module on \( M \), given by:
	\[
		\overline{a}u \coloneqq  au \quad \forall \; \overline{a} \in \qo{R}{I}, u\in M
	\]
\end{claim}
To check that this is well-defined, we need to check
\[
	\overline{a_1} = \overline{a_2} \implies a_1 u = a_2 u \quad \forall \; u\in M
\]
see that:
\[
	\overline{a_1} = \overline{a_2} \implies a_1 - a_2 \in I \implies (a_1 - a_2)u = 0 \implies a_1 u
	= a_2 u
\]
Suppose \( M \) is any left \( R \)-module, \( I \subseteq R \) be an \textcolor{blue}{left ideal},
define
\[
	\begin{aligned}
		IM & \coloneqq  \angl{\;\{au \; | \; a\in I, u \in M\} \; }                                     \\
		   & \xlongequal{\text{exer!}} \brac{\sum_{i=1}^r a_i u_i \; | \; r\geq 0, a_i \in I, u_i \in M
			\; \forall \; i}
	\end{aligned}
\]
Consider \( \overline{M} \coloneqq \qo{M}{IM} \), this satisfies:
\[
	\lambda \in I, \overline{u} \in \overline{M} \implies \lambda \overline{u} = \overline{\lambda u}
	= 0
\]
\begin{conclusion}
	If \( I \) is a two-sided ideal \( \implies \overline{M} \) has a natural structure of \( \qo{R}{I} \)-module.
\end{conclusion}

\begin{note}
	The submodules are the same (whether viewed over \( R \) or over \( \qo{R}{I} \)).
\end{note}

\begin{exercise}
	Show that we get in this way, a \textbf{\textcolor{red}{functor}}
	\[
		R-\underline{\text{mod}} \to \qo{R}{I}-\underline{\text{mod}}
	\]
	We need to check: if \( f: M\to N \) is a morphism of \( R \)-modules, then we get a morphism of
	\( \qo{R}{I} \)-modules
	\[
		\begin{aligned}
			\qo{M}{IM}   & \to \qo{N}{IN}          \\
			\overline{u} & \mapsto \overline{f(u)}
		\end{aligned}
	\]
\end{exercise}

\section{Free Modules}
It is difficult to studyy on modules where \( R \) is not a field, because in general there is no
basis for such kinds of modules. Free modules along with their basis along us to understand ideal of
rings in a larger context.

\begin{definition}[\textbf{Free Modules}]
	Fix \( R \) be ring, and left \( R \)-module \( M \). \( A \subseteq M \) is \underline{linearly
		independent} (over \( R \)) if for every relation
	\[
		\sum \lambda_i x_i = 0
	\]
	with \( \lambda_i \in R \) and distinct \( u_i \in A \) with all but \textbf{finitely many} \(
	\lambda_i = 0 \), \( \implies \lambda_i = 0 \; \forall \; i \).

	\( A \subseteq M \) is a \underline{basis} of \( M \) if \( A \) is linearly independent and \(
	\angl{A} = M \). \( M \) is \underline{free \( R \)-modules} if it has a basis.
\end{definition}

\begin{remark}
	Keep in mind, if \( R \ne \) field \( \implies \) ``\textbf{\textcolor{red}{most}}'' \( R \)
	modules are \textbf{not free}.
\end{remark}

\begin{eg}
	\leavevmode
	\begin{itemize}
		\item Suppose \( a\in R \backslash \{0\} \) and \( R \) be commutative ring, \( a \) is not a
		      unit. Then
		      \[
			      M = \qo{R}{(a)} \ne 0
		      \]
		      don't have a basis since every element \( u \in M \) sarisfies \( au = 0 \implies \) there is
		      no non-empty linearly independent subset \( \implies M \ne \) free \( R \)-modules.
		\item \( R \) is free module as a left \( R \)-module with basis being \( 1 \).
	\end{itemize}
\end{eg}

